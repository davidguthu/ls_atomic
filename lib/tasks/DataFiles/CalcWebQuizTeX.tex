deltax\\
\(\displaystyle \Delta x \)

limitdef\\
\(\displaystyle \frac{f(x + \Delta x) - f(x)}{\Delta x} \)

limitquizdirections\\
\(\displaystyle \text{Use the limti definition of the derivative to verify each derivative. Select the simplified algebraic expression that occurs before you take the li} \Delta x \text{ approaches } 0. \text{ In other words, to what does } \frac{f(x + \Delta x) - f(x)}{\Delta x} \text{ simplify, before you take the limit?} \)

q0111r1\\
\(\displaystyle (x/5)(3-x^2)^5 + C \)

q0111r2\\
\(\displaystyle -(1/10)(3-x^2)^5 + C \)

q0111r3\\
\(\displaystyle (1/5)(3-x^2)^5 + C \)

q0111r4\\
\(\displaystyle -(1/5)(3-x^2)^5 + C \)

q0111\\
\(\displaystyle \text{Evaluate } \int x(3-x^2)^4 \, dx. \)

q0112r1\\
\(\displaystyle -\cos^5 x + C \)

q0112r2\\
\(\displaystyle (1/5) \cos^5 x + C \)

q0112r3\\
\(\displaystyle -(1/5) \cos^5 x + C \)

q0112r4\\
\(\displaystyle (1/5) \cos^5 x \sin x + C \)

q0112\\
\(\displaystyle \text{Evaluate } \int \cos^4 x \sin x \, dx. \)

q0113r1\\
\(\displaystyle (2/3)(x+1)^{3/2} - 2(x+1)^{1/2} + C \)

q0113r2\\
\(\displaystyle (2/3)(x+1)^{3/2} + C \)

q0113r3\\
\(\displaystyle (2/3)(x+1)^{3/2} + (x+1)^{1/2} + C \)

q0113r4\\
\(\displaystyle (x+1)^{3/2} - (x+1)^{1/2} + C \)

q0113\\
\(\displaystyle \text{Evaluate the indefinite integral } \int \frac{x}{\sqrt{x+1}}\,dx. \)

q0114r1\\
\(\displaystyle 8/5 \)

q0114r2\\
\(\displaystyle 4/5 \)

q0114r3\\
\(\displaystyle 31/10 \)

q0114r4\\
\(\displaystyle 16/5 \)

q0114\\
\(\displaystyle \text{Evaluate the definite integral } \int_0^1 x(x^2 + 1)^4 \, dx. \)

q0115r1\\
\(\displaystyle -\ln 7 \)

q0115r2\\
\(\displaystyle 2 \ln 7 \)

q0115r3\\
\(\displaystyle (1/2) \ln 7 \)

q0115r4\\
\(\displaystyle \ln 7 \)

q0115\\
\(\displaystyle \text{Evaluate the definite integral } \int_1^3 \frac{6x}{3x^2 + 1} \, dx. \)

q0121r1\\
\(\displaystyle (2/3) (5x^2 + 6)^{3/2} + C \)

q0121r2\\
\(\displaystyle (1/15) (5x^2 + 6)^{3/2} + C \)

q0121r3\\
\(\displaystyle (1/10) (5x^2 + 6)^{3/2} + C \)

q0121r4\\
\(\displaystyle (x/10)(5x^2 + 6)^{3/2} + C \)

q0121\\
\(\displaystyle \text{Evaluate the indefinite integral } \int x(5x^2 + 6)^{1/2} \, dx. \)

q0122r1\\
\(\displaystyle \sin x (1 + \cos x)^{1/2} + C \)

q0122r2\\
\(\displaystyle (-1/2) (1 + \cos x)^{1/2} + C \)

q0122r3\\
\(\displaystyle (1 + \cos x)^{1/2} + C \)

q0122r4\\
\(\displaystyle -2(1 + \cos x)^{1/2} + C \)

q0122\\
\(\displaystyle \text{Evaluate the indefinite integral } \int \frac{\sin x}{\sqrt{1 + \cos x}}\,dx. \)

q0123r1\\
\(\displaystyle x^{16}/16 + C \)

q0123r2\\
\(\displaystyle \frac{(x^2 + 1)^8}{8} + C \)

q0123r3\\
\(\displaystyle \frac{(x^2 + 1)^8}{16} - \frac{(x^2 + 1)^7}{14} + C \)

q0123r4\\
\(\displaystyle \frac{(x^2 + 1)^8}{16} + \frac{(x^2 + 1)^7}{14} + C \)

q0123\\
\(\displaystyle \text{Evaluate the indefinite integral } \int x^3 (x^2 + 1)^6 \, dx. \)

q0124r1\\
\(\displaystyle (1/2)(2^{1/2} - 1) \)

q0124r2\\
\(\displaystyle (1/2)(2^{1/2} + 1) \)

q0124r3\\
\(\displaystyle 2^{1/2} - 1 \)

q0124r4\\
\(\displaystyle 2^{3/2} + 1 \)

q0124\\
\(\displaystyle \text{Evaluate the definite integral } \int_0^1 \frac{x^3}{(x^4 + 1)^{1/2}}\,dx. \)

q0125r1\\
\(\displaystyle 1 + 1/e \)

q0125r2\\
\(\displaystyle 1/(2e) \)

q0125r3\\
\(\displaystyle (1/2)(1 - 1/e) \)

q0125r4\\
\(\displaystyle 2e - 1 \)

q0125\\
\(\displaystyle \text{Evaluate the definite integral } \int_0^1 xe^{-x^2}\,dx. \)

q0131r1\\
\(\displaystyle x^3(x^4 + 1)^{1/2} + C \)

q0131r2\\
\(\displaystyle (x^4 + 1)^{1/2} + C \)

q0131r3\\
\(\displaystyle (1/6) (x^4 + 1)^{3/2} + C \)

q0131r4\\
\(\displaystyle (1/8)(x^4 + 1)^{1/2} + C \)

q0131\\
\(\displaystyle \text{Evaluate the indefinite integral } \int x^3 (x^4 + 1)^{1/2} \, dx. \)

q0132r1\\
\(\displaystyle (2/3) (1 + \sin x)^{3/2} + C \)

q0132r2\\
\(\displaystyle (1 + \sin x)^{3/2} + C \)

q0132r3\\
\(\displaystyle (-3/2) (1 + \sin x)^{3/2} + C \)

q0132r4\\
\(\displaystyle \cos x (1 + \sin x)^{3/2} + C \)

q0132\\
\(\displaystyle \text{Evaluate } \int (1 + \sin x)^{1/2} \cos x \, dx. \)

q0133r1\\
\(\displaystyle x^9/9 - x^8/8 + C \)

q0133r2\\
\(\displaystyle x^9/9 + C \)

q0133r3\\
\(\displaystyle (x+1)^9 - (x+1)^8 + C \)

q0133r4\\
\(\displaystyle (x+1)^9/9 - (x+1)^8/8 + C \)

q0133\\
\(\displaystyle \text{Evaluate } \int x(x + 1)^7 \, dx. \)

q0134r1\\
\(\displaystyle (8/7)2^{1/2} + 1 \)

q0134r2\\
\(\displaystyle (1/7)[8(2)^{1/2} - 1] \)

q0134r3\\
\(\displaystyle 4(2)^{1/2} - 1 \)

q0134r4\\
\(\displaystyle 8(2)^{1/2} + 1/7 \)

q0134\\
\(\displaystyle \text{Evaluate the definite integral } \int_0^1 x(x^2 + 1)^{5/2} \, dx. \)

q0135r1\\
\(\displaystyle 2/3 \)

q0135r2\\
\(\displaystyle 0 \)

q0135r3\\
\(\displaystyle 1/3 \)

q0135r4\\
\(\displaystyle 1 \)

q0135\\
\(\displaystyle \text{Evaluate the definite integral } \int_{-\pi/2}^{\pi/2} \sin^2 x \cos x \, dx. \)

q0710r1\\
\(\displaystyle f(x) = e^{-(x-2)^2} \)

q0710r2\\
\(\displaystyle f(x) = e^{-(x+2)^2} \)

q0710r3\\
\(\displaystyle f(x) = -2e^{-x^2} \)

q0710r4\\
\(\displaystyle f(x) = e^{-2x^2} \)

q0710\\
\(\displaystyle \text{What function has the following graph?} \)

q0711r1\\
\(\displaystyle (-\infty, \infty), (0, \infty) \)

q0711r2\\
\(\displaystyle (4, \infty), [0, \infty) \)

q0711r3\\
\(\displaystyle (4, \infty), (0, \infty) \)

q0711r4\\
\(\displaystyle [4, \infty), (-\infty, \infty) \)

q0711\\
\(\displaystyle \text{What are the domain and range, respectively, of } f(x) = \frac{1}{(x-4)^{1/2}}? \)

q0712r1\\
\(\displaystyle (x^5 - 5)^{1/2} \)

q0712r2\\
\(\displaystyle (x^9 - 5)^{1/2} \)

q0712r3\\
\(\displaystyle x^5 - 5 \)

q0712r4\\
\(\displaystyle (x^6 - 5)^{1/2} \)

q0712\\
\(\displaystyle \text{If } f(x) = (x^3 - 5)^{1/2}, \text{ what is } f(x^2)? \)

q0713r1\\
\(\displaystyle f(x) = |x| \)

q0713r2\\
\(\displaystyle f(x) = \ln x \)

q0713r3\\
\(\displaystyle f(x) = \frac{1}{x} + 3 \)

q0713r4\\
\(\displaystyle f(x) = x \sin x \)

q0713\\
\(\displaystyle \text{Which functions below are even? [Mark all that apply.]} \)

q0714r1\\
\(\displaystyle f(x) = \ln x \)

q0714r2\\
\(\displaystyle f(x) = x^{1/5} \)

q0714r3\\
\(\displaystyle f(x) = \frac{x}{x+3} \)

q0714r4\\
\(\displaystyle f(x) = \tan 3x \)

q0714\\
\(\displaystyle \text{Which functions are odd? [Mark all that apply.]} \)

q0715r1\\
\(\displaystyle \text{A} \)

q0715r2\\
\(\displaystyle \text{B} \)

q0715r3\\
\(\displaystyle \text{D} \)

q0715r4\\
\(\displaystyle \text{F} \)

q0715\\
\(\displaystyle f(x) = \cos 4x \)

q0716r1\\
\(\displaystyle \text{B} \)

q0716r2\\
\(\displaystyle \text{C} \)

q0716r3\\
\(\displaystyle \text{E} \)

q0716r4\\
\(\displaystyle \text{F} \)

q0716\\
\(\displaystyle f(x) = \cos(x/2) + 1 \)

q0717r1\\
\(\displaystyle \text{A} \)

q0717r2\\
\(\displaystyle \text{C} \)

q0717r3\\
\(\displaystyle \text{D} \)

q0717r4\\
\(\displaystyle \text{E} \)

q0717\\
\(\displaystyle f(x) = -\cos(x - \pi/2) \)

q0718r1\\
\(\displaystyle \text{B} \)

q0718r2\\
\(\displaystyle \text{C} \)

q0718r3\\
\(\displaystyle \text{D} \)

q0718r4\\
\(\displaystyle \text{F} \)

q0718\\
\(\displaystyle f(x) = 2\cos 2x \)

q0719r1\\
\(\displaystyle f(x) = -|x+2| + 1 \)

q0719r2\\
\(\displaystyle f(x) = -2|x| + 1 \)

q0719r3\\
\(\displaystyle f(x) = -2|x+2| + 1 \)

q0719r4\\
\(\displaystyle f(x) = -0.5|x+2| + 1 \)

q0719\\
\(\displaystyle \text{What function has the following graph?} \)

q0720r1\\
\(\displaystyle f(x) + -2e^{x^2} \)

q0720r2\\
\(\displaystyle f(x) = e^{-(x+2)^2} \)

q0720r3\\
\(\displaystyle f(x) = 2e^{(x+1)^2} \)

q0720r4\\
\(\displaystyle f(x) = e^{(x+2)^2} \)

q0720\\
\(\displaystyle \text{What function has the following graph?} \)

q0721r1\\
\(\displaystyle (-\infty, \infty), (0, 1/2] \)

q0721r2\\
\(\displaystyle (0,\infty), (0, \infty) \)

q0721r3\\
\(\displaystyle (-\infty, \infty), (0, \infty) \)

q0721r4\\
\(\displaystyle (-\infty, \infty), (0, 1/2) \)

q0721\\
\(\displaystyle \text{What are the domain and range, respectively, of } f(x) = \frac{1}{x^2 + 2}? \)

q0722r1\\
\(\displaystyle x + 1/4 \)

q0722r2\\
\(\displaystyle 1/x^2 + 4 \)

q0722r3\\
\(\displaystyle \frac{1}{x^2 + 4} \)

q0722r4\\
\(\displaystyle 1/x^2 + 1/4 \)

q0722\\
\(\displaystyle \text{If } f(x) = x^2 + 4, \text{ what is } f(1/x)? \)

q0723r1\\
\(\displaystyle f(x) = (x-1)^{1/2} \)

q0723r2\\
\(\displaystyle f(x) = 2x^4 - x^2 + 3 \)

q0723r3\\
\(\displaystyle f(x) = \cos x - \sin x \)

q0723r4\\
\(\displaystyle f(x) = e^{-x^2} \)

q0723\\
\(\displaystyle \text{Which functions below are even? [Mark all that apply.]} \)

q0724r1\\
\(\displaystyle f(x) = -1 \text{ for } x < 0, f(x) = 0 \text{ for } x = 0, f(x) = 1 \text{ for } x > 0 \)

q0724r2\\
\(\displaystyle f(x) = -\sin 5x \)

q0724r3\\
\(\displaystyle f(x) = x \sin x \)

q0724r4\\
\(\displaystyle f(x) = (x+3)^{1/2} \)

q0724\\
\(\displaystyle \text{Which functions below are odd? [Mark all that apply.]} \)

q0725r1\\
\(\displaystyle \text{A} \)

q0725r2\\
\(\displaystyle \text{D} \)

q0725r3\\
\(\displaystyle \text{E} \)

q0725r4\\
\(\displaystyle \text{F} \)

q0725\\
\(\displaystyle f(x) = 2|x+1| \)

q0726r1\\
\(\displaystyle \text{A} \)

q0726r2\\
\(\displaystyle \text{B} \)

q0726r3\\
\(\displaystyle \text{D} \)

q0726r4\\
\(\displaystyle \text{F} \)

q0726\\
\(\displaystyle f(x) = |x-2| - 1 \)

q0727r1\\
\(\displaystyle \text{C} \)

q0727r2\\
\(\displaystyle \text{D} \)

q0727r3\\
\(\displaystyle \text{E} \)

q0727r4\\
\(\displaystyle \text{F} \)

q0727\\
\(\displaystyle f(x) = -2|x| + 2 \)

q0728r1\\
\(\displaystyle \text{A} \)

q0728r2\\
\(\displaystyle \text{B} \)

q0728r3\\
\(\displaystyle \text{C} \)

q0728r4\\
\(\displaystyle \text{D} \)

q0728\\
\(\displaystyle f(x) = -|x-1| \)

q0729r1\\
\(\displaystyle f(x) = -2\sin(4x) \)

q0729r2\\
\(\displaystyle f(x) = -2\cos(4x) \)

q0729r3\\
\(\displaystyle f(x) = 2\cos(4x - \pi/2) \)

q0729r4\\
\(\displaystyle f(x) = 2\sin(2x - \pi) \)

q0729\\
\(\displaystyle \text{Which function has the following graph?} \)

q0730r1\\
\(\displaystyle f(x) = (1/3)\sin(x/2) \)

q0730r2\\
\(\displaystyle f(x) = (-1/3)\cos(x/2) \)

q0730r3\\
\(\displaystyle f(x) = (-1/3)\sin(x/2) \)

q0730r4\\
\(\displaystyle f(x) = (1/3)\cos(x + 1/2) \)

q0730\\
\(\displaystyle \text{What function has the following graph?} \)

q0731r1\\
\(\displaystyle (-\infty, \infty), (0, \infty) \)

q0731r2\\
\(\displaystyle (-\infty, 4) \cup (4, \infty), [0, \infty) \)

q0731r3\\
\(\displaystyle (-\infty, 4) \cup (4, \infty), (0, 1) \)

q0731r4\\
\(\displaystyle (-\infty, 4) \cup (4, \infty), (-\infty, 0) \cup (0, \infty) \)

q0731\\
\(\displaystyle \text{What are the domain and range, respectively, of } f(x) = \frac{1}{x-4}? \)

q0732r1\\
\(\displaystyle 12x^2 + 2x - 5 \)

q0732r2\\
\(\displaystyle 12x^2 + x - 5 \)

q0732r3\\
\(\displaystyle 6x^2 + 2x - 5 \)

q0732r4\\
\(\displaystyle 6x^2 - 9x \)

q0732\\
\(\displaystyle \text{If } f(x) = 3x^2 + x - 5, \text{ what is } f(2x)? \)

q0733r1\\
\(\displaystyle f(x) = (x^2 + 2)^{1/2} \)

q0733r2\\
\(\displaystyle f(x) = (x-3)^2 \)

q0733r3\\
\(\displaystyle f(x) = 7 \)

q0733r4\\
\(\displaystyle f(x) = x^4 + 2x - 1 \)

q0733\\
\(\displaystyle \text{Which functions below are even? [Mark all that apply.]} \)

q0734r1\\
\(\displaystyle f(x) = \frac{x}{x + 3} \)

q0734r2\\
\(\displaystyle f(x) = x^2 \cos x \)

q0734r3\\
\(\displaystyle f(x) = (4x)^{1/3} \)

q0734r4\\
\(\displaystyle f(x) = x^7 - x^5 + x^3 - x \)

q0734\\
\(\displaystyle \text{Which functions below are odd? [Mark all that apply.]} \)

q0735r1\\
\(\displaystyle \text{B} \)

q0735r2\\
\(\displaystyle \text{D} \)

q0735r3\\
\(\displaystyle \text{E} \)

q0735r4\\
\(\displaystyle \text{F} \)

q0735\\
\(\displaystyle f(x) = 2e^{-x^2} \)

q0736r1\\
\(\displaystyle \text{A} \)

q0736r2\\
\(\displaystyle \text{B} \)

q0736r3\\
\(\displaystyle \text{D} \)

q0736r4\\
\(\displaystyle \text{E} \)

q0736\\
\(\displaystyle f(x) = 2e^{-(x+1)^2} \)

q0737r1\\
\(\displaystyle \text{A} \)

q0737r2\\
\(\displaystyle \text{B} \)

q0737r3\\
\(\displaystyle \text{C} \)

q0737r4\\
\(\displaystyle \text{D} \)

q0737\\
\(\displaystyle f(x) = -2e^{-(x+1)^2} \)

q0738r1\\
\(\displaystyle \text{C} \)

q0738r2\\
\(\displaystyle \text{D} \)

q0738r3\\
\(\displaystyle \text{E} \)

q0738r4\\
\(\displaystyle \text{F} \)

q0738\\
\(\displaystyle f(x) = e^{-(x-1)^2} \)

q0739r1\\
\(\displaystyle f(x) = 2|x| + 1 \)

q0739r2\\
\(\displaystyle f(x) = 0.5|x-2| - 1 \)

q0739r3\\
\(\displaystyle f(x) = 0.5|x-2| + 1 \)

q0739r4\\
\(\displaystyle f(x) = 2|x-2| + 1 \)

q0739\\
\(\displaystyle \text{What function has the following graph?} \)

q1211r1\\
\(\displaystyle (x/3) e^{3x} - (1/9)e^{3x} + C \)

q1211r2\\
\(\displaystyle (1/3)e^{3x} - (x/9)e^{3x} + C \)

q1211r3\\
\(\displaystyle (x/3)e^{3x} + (x/9)e^{3x} + C \)

q1211r4\\
\(\displaystyle (1/3)e^{3x} - (1/9)e^{3x} + C \)

q1211\\
\(\displaystyle \text{Evaluate } \int x e^{3x}\,dx. \)

q1212r1\\
\(\displaystyle x^2 - x\sin x + \cos x + C \)

q1212r2\\
\(\displaystyle 2x^2 - x\sin x - \cos x + C \)

q1212r3\\
\(\displaystyle (1/2)x^2 + x \sin x + \cos x + C \)

q1212r4\\
\(\displaystyle (1/2)x^2 + \sin x - \cos x + C \)

q1212\\
\(\displaystyle \text{Evaluate } \int x(1 + \cos x)\,dx \)

q1213r1\\
\(\displaystyle x \sin x - 2\cos x - \sin x + C \)

q1213r2\\
\(\displaystyle x^2 \sin x + 2x \cos x - 2 \sin x + C \)

q1213r3\\
\(\displaystyle x^2 \sin x - 2x\cos x + 2\sin x + C \)

q1213r4\\
\(\displaystyle x \sin x + 2\cos x - 2\sin x + C \)

q1213\\
\(\displaystyle \text{Evaluate } \int x^2 \cos x \, dx \)

q1214r1\\
\(\displaystyle (1/3)e^3 - 1/9 \)

q1214r2\\
\(\displaystyle (2/9)e^3 \)

q1214r3\\
\(\displaystyle (1/3)e^3 - 1/3 \)

q1214r4\\
\(\displaystyle (2/9)e^3 + 1/9 \)

q1214\\
\(\displaystyle \text{Evaluate the definite integral } \int_1^e x^2 \ln x \, dx. \)

q1215r1\\
\(\displaystyle -\pi/24 + (3^{1/2} + 2)/4 \)

q1215r2\\
\(\displaystyle \pi/24 + (3^{1/2} - 2)/4 \)

q1215r3\\
\(\displaystyle \pi/24 \)

q1215r4\\
\(\displaystyle (2-3^{1/2})/2 \)

q1215\\
\(\displaystyle \text{Evaluate the definite integral } \int_0^{1/4} \sin^{-1} (2x)\,dx. \text{ Recall that the derivative of } \sin^{-1} u \text{ is } \frac{1}{(1-u^2)^{1/2}} \)

q1221r1\\
\(\displaystyle x \ln x + x + C \)

q1221r2\\
\(\displaystyle \ln x - x + C \)

q1221r3\\
\(\displaystyle x \ln x - x + C \)

q1221r4\\
\(\displaystyle 2\ln x - x + C \)

q1221\\
\(\displaystyle \text{Evaluate } \int \ln x\,dx \)

q1222r1\\
\(\displaystyle (-x/2)\cos 2x + (1/4) \sin 2x + C \)

q1222r2\\
\(\displaystyle (1/2) \cos 2x + (x/4)\sin 2x + C \)

q1222r3\\
\(\displaystyle (x/2)\sin 2x  - (1/2)\cos 2x + C \)

q1222r4\\
\(\displaystyle (x/4)\cos 2x - (3/4) \sin 2x + C \)

q1222\\
\(\displaystyle \text{Evaluate } \int x \sin 2x \, dx. \)

q1223r1\\
\(\displaystyle \frac{ (\sin x + \cos x)e^x}{2} \)

q1223r2\\
\(\displaystyle \frac{(\sin x + \cos x)e^x}{4} \)

q1223r3\\
\(\displaystyle (\sin x - \cos x)e^x \)

q1223r4\\
\(\displaystyle \frac{\sin x - \cos x)e^x}{2} \)

q1223\\
\(\displaystyle \text{Evaluate } \int e^x \sin x \, dx. \)

q1224r1\\
\(\displaystyle \frac{6}{25e^5} \)

q1224r2\\
\(\displaystyle 1/25 - \frac{6}{25e^5} \)

q1224r3\\
\(\displaystyle \frac{-1}{5e^5} \)

q1224r4\\
\(\displaystyle \frac{1}{25e^5} - \frac{6}{25} \)

q1224\\
\(\displaystyle \text{Evaluate the definite integral } \int_0^1 xe^{-5x} \, dx. \)

q1225r1\\
\(\displaystyle \pi/4 + \ln 2 \)

q1225r2\\
\(\displaystyle (1/4) \ln 2 \)

q1225r3\\
\(\displaystyle -\pi/8 \)

q1225r4\\
\(\displaystyle \pi/8 - (1/4) \ln 2 \)

q1225\\
\(\displaystyle \text{Evaluate the definite integral } \int_0^1 x \tan^{-1} x^2 \, dx. \text{ Recall that the derivative of } \tan^{-1} u \text{ us } \frac{1}{1+u^2}. \)

q1231r1\\
\(\displaystyle (x^2) \ln x + x^2/2 + C \)

q1231r2\\
\(\displaystyle (x^2 / 2) \ln x - x^2 /4 + C \)

q1231r3\\
\(\displaystyle x \ln x - x^2/2 + C \)

q1231r4\\
\(\displaystyle (2x^2) \ln x - x^2/2 + C \)

q1231\\
\(\displaystyle \text{Evaluate } \int x\ln x \, dx. \)

q1232r1\\
\(\displaystyle x \tan x - \ln |\cos x| + C \)

q1232r2\\
\(\displaystyle \tan x + \ln|\sin x| + C \)

q1232r3\\
\(\displaystyle x\sec x + 2\ln|\cos x| + C \)

q1232r4\\
\(\displaystyle x \tan x + \ln|\cos x| + C \)

q1232\\
\(\displaystyle \text{Evaluate } \int x \sec^2 x \, dx. \)

q1233r1\\
\(\displaystyle (x/2)[\cos(\ln x) + \sin(\ln x)] + C \)

q1233r2\\
\(\displaystyle (1/2)[\cos(\ln x) - \sin(\ln x)] + C \)

q1233r3\\
\(\displaystyle x[2 \cos (\ln x) - \sin(\ln x)] + C \)

q1233r4\\
\(\displaystyle (x/4)[\cos(\ln x) + 2\sin(\ln x)] + C \)

q1233\\
\(\displaystyle \text{Evaluate } \int \cos(\ln x)\,dx. \)

q1234r1\\
\(\displaystyle 1 - 3/e \)

q1234r2\\
\(\displaystyle 2 + 5/e \)

q1234r3\\
\(\displaystyle 2 - 5/e \)

q1234r4\\
\(\displaystyle 4 + e \)

q1234\\
\(\displaystyle \text{Evaluate the definite integral } \int_0^1 x^2 e^{-x}\,dx. \)

q1235r1\\
\(\displaystyle 2/\pi \)

q1235r2\\
\(\displaystyle 2/\pi - 4/\pi^2 \)

q1235r3\\
\(\displaystyle 4/\pi \)

q1235r4\\
\(\displaystyle 2/\pi^2 \)

q1235\\
\(\displaystyle \text{Evaluate the definite integral } \int_0^1 2x \sin(\pi x)\,dx. \)

q1511r1\\
\(\displaystyle \frac{2x}{(1-x)^2} \)

q1511r2\\
\(\displaystyle \frac{2x-x^2}{(1-x)^2} \)

q1511r3\\
\(\displaystyle \frac{x^2}{(1-x)^2} \)

q1511r4\\
\(\displaystyle \frac{x-2x^2}{(1-x)^2} \)

q1511\\
\(\displaystyle \text{Differentiate } f(x) = \frac{x^2}{1-x} \)

q1512r1\\
\(\displaystyle \frac{-5}{3x-2} \)

q1512r2\\
\(\displaystyle \frac{3-2x}{(3x-2)^2} \)

q1512r3\\
\(\displaystyle \frac{2}{(3x-2)^2} \)

q1512r4\\
\(\displaystyle \frac{5}{(3x-2)^2} \)

q1512\\
\(\displaystyle \text{Differentiate } f(x) = \frac{2x-3}{3x-2}. \)

q1513r1\\
\(\displaystyle \frac{1}{x(x+1)^2} \)

q1513r2\\
\(\displaystyle \frac{-\ln|x+1|}{x+1} \)

q1513r3\\
\(\displaystyle \frac{1-\ln|x+1|}{(x+1)^2} \)

q1513r4\\
\(\displaystyle \frac{\ln|x+1|}{(x+1)^2} \)

q1513\\
\(\displaystyle \text{Differentiate } f(x) = \frac{\ln|x+1|}{x+1}. \)

q1514r1\\
\(\displaystyle 0 \)

q1514r2\\
\(\displaystyle f'(1) - f(1) \)

q1514r3\\
\(\displaystyle f'(1) \)

q1514r4\\
\(\displaystyle f'(1) + 2f(1) \)

q1514\\
\(\displaystyle \text{Find the derivative of } \frac{f(x)}{x} \text{ at } x =1. \)

q1515r1\\
\(\displaystyle \frac{6x^2 - 2}{(x^2+1)^3} \)

q1515r2\\
\(\displaystyle \frac{-2x}{(x^2+1)^3} \)

q1515r3\\
\(\displaystyle \frac{2x-1}{(x^2+1)^3} \)

q1515r4\\
\(\displaystyle \frac{4x^2 - 3}{(x^2+1)^4} \)

q1515\\
\(\displaystyle \text{Find the second derivative of } \frac{1}{x^2+1}. \)

q1521r1\\
\(\displaystyle \frac{2x-1}{(2x+1)^2} \)

q1521r2\\
\(\displaystyle \frac{3}{2x+1} \)

q1521r3\\
\(\displaystyle \frac{x}{(2x+1)^2} \)

q1521r4\\
\(\displaystyle \frac{3}{(2x+1)^2} \)

q1521\\
\(\displaystyle \text{Differentiate } f(x) = \frac{3x}{2x+1} \)

q1522r1\\
\(\displaystyle \frac{x^4 - 5x^2 + 2x - 1}{(x^2-1)^2} \)

q1522r2\\
\(\displaystyle \frac{3x^4 - 2x + 3}{(x^2-1)^2} \)

q1522r3\\
\(\displaystyle \frac{x^4 - 6x^2 - 4x - 3}{(x^2-1)^2} \)

q1522r4\\
\(\displaystyle \frac{2x^3 - 4x^2 + 1}{(x^2-1)^2} \)

q1522\\
\(\displaystyle \text{Differentiate } f(x) = \frac{x^3 + 3x + 2}{x^2-1}. \)

q1523r1\\
\(\displaystyle \frac{x\cos x - \sin x}{x^2} \)

q1523r2\\
\(\displaystyle \cos x \)

q1523r3\\
\(\displaystyle \frac{x \sin x - \cos x}{x} \)

q1523r4\\
\(\displaystyle \frac{x^2 \sin x - \sin x}{x^2} \)

q1523\\
\(\displaystyle \text{Differentiate } f(x) = \frac{\sin x}{x}. \)

q1524r1\\
\(\displaystyle \frac{ab - b^2}{c+d} \)

q1524r2\\
\(\displaystyle \frac{ad + bc}{d} \)

q1524r3\\
\(\displaystyle \frac{ad - bc}{d^2} \)

q1524r4\\
\(\displaystyle \frac{a^2 + b^2}{(c+d)^2} \)

q1524\\
\(\displaystyle \text{If } f(x) = \frac{ax + b}{cx + d}, \text{ what is } f'(0)? \)

q1525r1\\
\(\displaystyle \frac{2}{x^3} \)

q1525r2\\
\(\displaystyle \frac{-6}{x^3} \)

q1525r3\\
\(\displaystyle \frac{x+3}{x^3} \)

q1525r4\\
\(\displaystyle \frac{6}{x^2} \)

q1525\\
\(\displaystyle \text{What is the second derivative of } \frac{x^2 - 3}{x}? \)

q1531r1\\
\(\displaystyle \frac{-1}{(x+1)^2} \)

q1531r2\\
\(\displaystyle \frac{x}{(x+1)^2} \)

q1531r3\\
\(\displaystyle \frac{1}{(x+1)^2} \)

q1531r4\\
\(\displaystyle \frac{2x-1}{(x+1)^2} \)

q1531\\
\(\displaystyle \text{Differentiate } f(x) = \frac{x}{x+1} \)

q1532r1\\
\(\displaystyle \frac{2(x^2 + 3x + 1)}{(2x+3)^2} \)

q1532r2\\
\(\displaystyle \frac{2x^2 - x + 1}{(2x+3)^2} \)

q1532r3\\
\(\displaystyle \frac{x^2 - 3x + 2}{(2x+3)^2} \)

q1532r4\\
\(\displaystyle \frac{x-1}{(2x+3)^2} \)

q1532\\
\(\displaystyle \text{Differentiate } f(x) = \frac{x^2 - 1}{2x+3} \)

q1533r1\\
\(\displaystyle \frac{ae^{ax} - be^{bx}}{(e^{ax} + e^{bx})^2} \)

q1533r2\\
\(\displaystyle \frac{2(a+b)e^{(a+b)x}}{e^{ax} + e^{bx}} \)

q1533r3\\
\(\displaystyle \frac{e^{abx}}{(e^{ax} + e^{bx})^2} \)

q1533r4\\
\(\displaystyle \frac{2(a-b)e^{(a+b)x}}{(e^{ax} + e^{bx})^2} \)

q1533\\
\(\displaystyle \text{Differentiate } f(x) = \frac{e^{ax} - e^{bx}}{e^{ax} + e^{bx}} \)

q1534r1\\
\(\displaystyle -1 \)

q1534r2\\
\(\displaystyle 1/2 \)

q1534r3\\
\(\displaystyle 0 \)

q1534r4\\
\(\displaystyle 1 \)

q1534\\
\(\displaystyle \text{If } f(x) = \frac{1-x^2}{1+x^2}, \text{ what is} f'(0)? \)

q1535r1\\
\(\displaystyle -7/16 \)

q1535r2\\
\(\displaystyle -23/16 \)

q1535r3\\
\(\displaystyle 13/16 \)

q1535r4\\
\(\displaystyle 2 \)

q1535\\
\(\displaystyle \text{If } f(4) = 3 \text{ and } f'(4) = -5, \text{ find } g'(4) \text{ if } g(x) = \frac{f(x)}{x}. \)

q1811r1\\
\(\displaystyle 1 \)

q1811r2\\
\(\displaystyle 0 \)

q1811r3\\
\(\displaystyle \Delta x \)

q1811r4\\
\(\displaystyle 2\Delta x \)

q1811\\
\(\displaystyle \frac{d}{dx} [3] = 0. \)

q1812r1\\
\(\displaystyle -1 \)

q1812r2\\
\(\displaystyle -1 + \Delta x \)

q1812r3\\
\(\displaystyle -1 - \Delta x \)

q1812r4\\
\(\displaystyle -1 + 2 \Delta x \)

q1812\\
\(\displaystyle \frac{d}{dx} [2-x] = -1. \)

q1813r1\\
\(\displaystyle 4x \)

q1813r2\\
\(\displaystyle 4x + \Delta x \)

q1813r3\\
\(\displaystyle 4x + 2\Delta x \)

q1813r4\\
\(\displaystyle 4x + 4\Delta x \)

q1813\\
\(\displaystyle \frac{d}{dx} [2x^2 - 3] = 4x. \)

q1814r1\\
\(\displaystyle \frac{-1}{(x+3)^2} \)

q1814r2\\
\(\displaystyle \frac{-\Delta x}{(x+\Delta x + 3)^2} \)

q1814r3\\
\(\displaystyle \frac{1}{(x+3)(x+\Delta x + 3)} \)

q1814r4\\
\(\displaystyle \frac{-1}{(x+3)(x + \Delta x + 3)} \)

q1814\\
\(\displaystyle \frac{d}{dx} \left [ \frac{1}{x+3} \right ] = \frac{-1}{(x+3)^2} \)

q1815r1\\
\(\displaystyle \frac{1}{(5x + \Delta x)^{1/2} + (5x)^{1/2}} \)

q1815r2\\
\(\displaystyle \frac{5}{(5x + 5\Delta x)^{1/2} + (5x)^{1/2}} \)

q1815r3\\
\(\displaystyle \frac{5}{(5x + 5\Delta x)^{1/2}} \)

q1815r4\\
\(\displaystyle \frac{5}{(5x + 5\Delta x)^{1/2} - (5x)^{1/2}} \)

q1815\\
\(\displaystyle \frac{d}{dx} [(5x)^{1/2}] = \frac{5}{2(5x)^{1/2}} \)

q1821r1\\
\(\displaystyle 1 \)

q1821r2\\
\(\displaystyle \Delta x \)

q1821r3\\
\(\displaystyle 0 \)

q1821r4\\
\(\displaystyle 2\Delta x \)

q1821\\
\(\displaystyle \frac{d}{dx} [-5] = 0. \)

q1822r1\\
\(\displaystyle 11 + 2\Delta x \)

q1822r2\\
\(\displaystyle 11 \)

q1822r3\\
\(\displaystyle 11 + \Delta x \)

q1822r4\\
\(\displaystyle 11 - \Delta x \)

q1822\\
\(\displaystyle \frac{d}{dx}[11x + 8] = 11. \)

q1823r1\\
\(\displaystyle -6x - 6\Delta x \)

q1823r2\\
\(\displaystyle -6x \)

q1823r3\\
\(\displaystyle -6x + 3\Delta x \)

q1823r4\\
\(\displaystyle -6x - 3\Delta x \)

q1823\\
\(\displaystyle \frac{d}{dx} [1 - 3x^2] = -6x. \)

q1824r1\\
\(\displaystyle \frac{-1}{(x-5)^2} \)

q1824r2\\
\(\displaystyle \frac{-1}{(x-5)(x + \Delta x - 5)} \)

q1824r3\\
\(\displaystyle \frac{-1}{(x+\Delta x - 5)^2} \)

q1824r4\\
\(\displaystyle \frac{1}{(x-5)(x + \Delta x - 5)} \)

q1824\\
\(\displaystyle \frac{d}{dx} \left [ \frac{1}{x-5} \right ] = \frac{-1}{(x-5)^2} \)

q1825r1\\
\(\displaystyle \frac{2}{(2x+2\Delta x)^{1/2} + (2x)^{1/2}} \)

q1825r2\\
\(\displaystyle \frac{1}{(2x+2\Delta x)^{1/2} + (2x)^{1/2}} \)

q1825r3\\
\(\displaystyle \frac{\Delta x}{(2x+2\Delta x)^{1/2} + (2x)^{1/2}} \)

q1825r4\\
\(\displaystyle \frac{2}{(2x+2\Delta x)^{1/2}} \)

q1825\\
\(\displaystyle \frac{d}{dx} [ (2x)^{1/2} ] = \frac{1}{(2x)^{1/2}} \)

q1831r1\\
\(\displaystyle 1 \)

q1831r2\\
\(\displaystyle \Delta x \)

q1831r3\\
\(\displaystyle 2\Delta x \)

q1831r4\\
\(\displaystyle 0 \)

q1831\\
\(\displaystyle \frac{d}{dx}[7] = 0. \)

q1832r1\\
\(\displaystyle -4 + \Delta x \)

q1832r2\\
\(\displaystyle -4 - \Delta x \)

q1832r3\\
\(\displaystyle -4 \)

q1832r4\\
\(\displaystyle -4 + 2\Delta x \)

q1832\\
\(\displaystyle \frac{d}{dx}[3-4x] = -4. \)

q1833r1\\
\(\displaystyle 10x \)

q1833r2\\
\(\displaystyle 10x + 5\Delta x \)

q1833r3\\
\(\displaystyle 10x + \Delta x \)

q1833r4\\
\(\displaystyle 10x + 10\Delta x \)

q1833\\
\(\displaystyle \frac{d}{dx}[5x^2 - 2] = 10x. \)

q1834r1\\
\(\displaystyle \frac{-2\Delta x}{(2x-1)^2} \)

q1834r2\\
\(\displaystyle \frac{-2}{(2x-1)(2x + 2\Delta x - 1)} \)

q1834r3\\
\(\displaystyle \frac{2}{(2x + 2\Delta x - 1)^2} \)

q1834r4\\
\(\displaystyle \frac{2}{(2x-1)(2x + 2\Delta x - 1)} \)

q1834\\
\(\displaystyle \frac{d}{dx} \left [ \frac{1}{2x-1} \right ] = \frac{-2}{(2x-1)^2} \)

q1835r1\\
\(\displaystyle \frac{1}{(3x + 3\Delta x)^{1/2} + (3x)^{1/2}} \)

q1835r2\\
\(\displaystyle \frac{3}{(3x + 3\Delta x)^{1/2}} \)

q1835r3\\
\(\displaystyle \frac{3}{(3x + 3\Delta x)^{1/2} - (3x)^{1/2}} \)

q1835r4\\
\(\displaystyle \frac{3}{(3x + 3\Delta x)^{1/2} + (3x)^{1/2}} \)

q1835\\
\(\displaystyle \frac{d}{dx} [ (3x)^{1/2}] = \frac{3}{2(3x)^{1/2}} \)

q2211r1\\
\(\displaystyle \infty \)

q2211r2\\
\(\displaystyle -1 \)

q2211r3\\
\(\displaystyle 0 \)

q2211r4\\
\(\displaystyle 1 \)

q2211\\
\(\displaystyle \lim_{x \rightarrow 0} \frac{xe^x}{1-e^x} \)

q2212r1\\
\(\displaystyle \infty \)

q2212r2\\
\(\displaystyle 1 \)

q2212r3\\
\(\displaystyle 0 \)

q2212r4\\
\(\displaystyle -\infty \)

q2212\\
\(\displaystyle \lim_{x \rightarrow \infty} \frac{\ln(x)}{x} \)

q2213r1\\
\(\displaystyle -\infty \)

q2213r2\\
\(\displaystyle 0 \)

q2213r3\\
\(\displaystyle 3 \)

q2213r4\\
\(\displaystyle \infty \)

q2213\\
\(\displaystyle \lim_{x \rightarrow \infty} \frac{x^3}{e^{-x}} \)

q2214r1\\
\(\displaystyle \infty \)

q2214r2\\
\(\displaystyle 1 \)

q2214r3\\
\(\displaystyle 0 \)

q2214r4\\
\(\displaystyle 1/2 \)

q2214\\
\(\displaystyle \lim_{x \rightarrow 0^+} \frac{\sin(x)}{x^2} \)

q2215r1\\
\(\displaystyle 0 \)

q2215r2\\
\(\displaystyle 1 \)

q2215r3\\
\(\displaystyle \infty \)

q2215r4\\
\(\displaystyle e \)

q2215\\
\(\displaystyle \lim_{x \rightarrow \infty} (1+1/x^2)^x \)

q2216r1\\
\(\displaystyle 1/10 \)

q2216r2\\
\(\displaystyle 0 \)

q2216r3\\
\(\displaystyle \infty \)

q2216r4\\
\(\displaystyle 1/6 \)

q2216\\
\(\displaystyle \lim_{x \rightarrow 2} \frac{x-1}{x^3+2} \)

q2217r1\\
\(\displaystyle \infty \)

q2217r2\\
\(\displaystyle 1 \)

q2217r3\\
\(\displaystyle 0 \)

q2217r4\\
\(\displaystyle -1 \)

q2217\\
\(\displaystyle \lim_{x \rightarrow 0} \frac{e^x - 1}{\sin(x)} \)

q2218r1\\
\(\displaystyle 1 \)

q2218r2\\
\(\displaystyle \infty \)

q2218r3\\
\(\displaystyle 0 \)

q2218r4\\
\(\displaystyle -1 \)

q2218\\
\(\displaystyle \lim_{x \rightarrow \infty} \frac{x^3}{1-x^3} \)

q2221r1\\
\(\displaystyle 0 \)

q2221r2\\
\(\displaystyle 1/3 \)

q2221r3\\
\(\displaystyle \infty \)

q2221r4\\
\(\displaystyle -1/3 \)

q2221\\
\(\displaystyle \lim_{x \rightarrow 0} \frac{1-\cos(x)}{3x} \)

q2222r1\\
\(\displaystyle \infty \)

q2222r2\\
\(\displaystyle 3 \)

q2222r3\\
\(\displaystyle 0 \)

q2222r4\\
\(\displaystyle 1 \)

q2222\\
\(\displaystyle \lim_{x \rightarrow 0} \frac{x^2 - 5x + 3}{x^2 - 2x + 1} \)

q2223r1\\
\(\displaystyle -\infty \)

q2223r2\\
\(\displaystyle 0 \)

q2223r3\\
\(\displaystyle 1 \)

q2223r4\\
\(\displaystyle \infty \)

q2223\\
\(\displaystyle \lim_{x \rightarrow \infty} \frac{x \ln (x)}{x + \ln(x)} \)

q2224r1\\
\(\displaystyle e^2 \)

q2224r2\\
\(\displaystyle e \)

q2224r3\\
\(\displaystyle \infty \)

q2224r4\\
\(\displaystyle 1 \)

q2224\\
\(\displaystyle \lim_{x \rightarrow \infty} (1+1/x)^{x^2} \)

q2225r1\\
\(\displaystyle 0 \)

q2225r2\\
\(\displaystyle \infty \)

q2225r3\\
\(\displaystyle 3 \)

q2225r4\\
\(\displaystyle 3/2 \)

q2225\\
\(\displaystyle \lim_{x \rightarrow \infty} \frac{e^{3x}}{x^2} \)

q2226r1\\
\(\displaystyle 1/9 \)

q2226r2\\
\(\displaystyle 0 \)

q2226r3\\
\(\displaystyle 1/5 \)

q2226r4\\
\(\displaystyle 3/4 \)

q2226\\
\(\displaystyle \lim_{x \rightarrow 3} \frac{x-3}{3x^2 - 13x + 12} \)

q2227r1\\
\(\displaystyle 1 \)

q2227r2\\
\(\displaystyle \infty \)

q2227r3\\
\(\displaystyle -\infty \)

q2227r4\\
\(\displaystyle 0 \)

q2227\\
\(\displaystyle \lim_{x \rightarrow -\infty} \frac{\cos(1/x)}{x} \)

q2228r1\\
\(\displaystyle 0 \)

q2228r2\\
\(\displaystyle \infty \)

q2228r3\\
\(\displaystyle 1 \)

q2228r4\\
\(\displaystyle -\infty \)

q2228\\
\(\displaystyle \lim_{x \rightarrow 0} \frac{\sin^2 (x)}{x} \)

q2231r1\\
\(\displaystyle 0 \)

q2231r2\\
\(\displaystyle -\infty \)

q2231r3\\
\(\displaystyle \infty \)

q2231r4\\
\(\displaystyle 1/6 \)

q2231\\
\(\displaystyle \lim_{x \rightarrow 0^+} \frac{\cos(x) - 1}{x^3} \)

q2232r1\\
\(\displaystyle 3/2 \)

q2232r2\\
\(\displaystyle 0 \)

q2232r3\\
\(\displaystyle -\infty \)

q2232r4\\
\(\displaystyle \infty \)

q2232\\
\(\displaystyle \lim_{x \rightarrow \infty} \frac{x^3 - 1}{2-x} \)

q2233r1\\
\(\displaystyle \infty \)

q2233r2\\
\(\displaystyle 0 \)

q2233r3\\
\(\displaystyle 1/2 \)

q2233r4\\
\(\displaystyle -1/2 \)

q2233\\
\(\displaystyle \lim_{x \rightarrow \pi/2} \frac{\cos(x)}{\sin(2x)} \)

q2234r1\\
\(\displaystyle 0 \)

q2234r2\\
\(\displaystyle 1 \)

q2234r3\\
\(\displaystyle \infty \)

q2234r4\\
\(\displaystyle -\infty \)

q2234\\
\(\displaystyle \lim_{x \rightarrow 0^+} \frac{x}{\ln(x)} \)

q2235r1\\
\(\displaystyle 0 \)

q2235r2\\
\(\displaystyle 1 \)

q2235r3\\
\(\displaystyle 1/2 \)

q2235r4\\
\(\displaystyle \infty \)

q2235\\
\(\displaystyle \lim_{x \rightarrow \infty} \left [ \frac{1+x^2}{2x} \right ] ^{1/2} \)

q2236r1\\
\(\displaystyle e^2 \)

q2236r2\\
\(\displaystyle \infty \)

q2236r3\\
\(\displaystyle 1 \)

q2236r4\\
\(\displaystyle 2 \)

q2236\\
\(\displaystyle \lim_{x \rightarrow \infty} (1+2x)^{-3/x} \)

q2237r1\\
\(\displaystyle 2/3 \)

q2237r2\\
\(\displaystyle 1 \)

q2237r3\\
\(\displaystyle \infty \)

q2237r4\\
\(\displaystyle 0 \)

q2237\\
\(\displaystyle \lim_{x \rightarrow 0} \frac{1-e^{-2x}}{x^2 + 3x} \)

q2238r1\\
\(\displaystyle 0 \)

q2238r2\\
\(\displaystyle 1/2 \)

q2238r3\\
\(\displaystyle 2 \)

q2238r4\\
\(\displaystyle \infty \)

q2238\\
\(\displaystyle \lim_{x \rightarrow 1} \frac{x^2 - x + 3}{x + 5} \)

q2710r1\\
\(\displaystyle -1 \)

q2710r2\\
\(\displaystyle 0 \)

q2710r3\\
\(\displaystyle 2 \)

q2710r4\\
\(\displaystyle \text{Does not exist} \)

q2710\\
\(\displaystyle \text{Let } f(x) = 3x - 1 \text{ for } x < 0, f(x) = 2 \text{ for } x = 0, f(x)= -\sqrt{1-x} \text{ for } x > 0. \text{ Find } \lim_{x \rightarrow 0} f(x). \)

q2715r1\\
\(\displaystyle 0 \)

q2715r2\\
\(\displaystyle 1/10 \)

q2715r3\\
\(\displaystyle 1 \)

q2715r4\\
\(\displaystyle \text{Does not exist} \)

q2715\\
\(\displaystyle \text{Evaluate } \lim_{x \rightarrow 0^+} e^{1/x} \)

q2716r1\\
\(\displaystyle 0 \)

q2716r2\\
\(\displaystyle 1/10 \)

q2716r3\\
\(\displaystyle 1 \)

q2716r4\\
\(\displaystyle \text{Does not exist} \)

q2716\\
\(\displaystyle \text{Evaluate } \lim_{x \rightarrow 5} \frac{x-5}{x^2-25} \)

q2717r1\\
\(\displaystyle 0 \)

q2717r2\\
\(\displaystyle 1/10 \)

q2717r3\\
\(\displaystyle 1 \)

q2717r4\\
\(\displaystyle \text{Does not exist} \)

q2717\\
\(\displaystyle \text{Evaluate } \lim_{x \rightarrow \infty} \sin \left ( \frac{1}{x} \right ) \)

q2718r1\\
\(\displaystyle -1 \)

q2718r2\\
\(\displaystyle 0 \)

q2718r3\\
\(\displaystyle 2 \)

q2718r4\\
\(\displaystyle \text{Does not exist} \)

q2718\\
\(\displaystyle \text{Let } f(x) = 3x - 1 \text{ for } x < 0, f(x) = 2 \text{ for } x = 0, f(x) = -\sqrt{1-x} \text{ for } x > 0. \text{ Find } \lim_{x \rightarrow 0^-} f(x). \)

q2719r1\\
\(\displaystyle -1 \)

q2719r2\\
\(\displaystyle 0 \)

q2719r3\\
\(\displaystyle 2 \)

q2719r4\\
\(\displaystyle \text{Does not exist} \)

q2719\\
\(\displaystyle \text{Let } f(x) = 3x - 1 \text{ for } x < 0, f(x) = 2 \text{ for } x = 0, f(x)= -\sqrt{1-x} \text{ for } x > 0. \text{ Find } \lim_{x \rightarrow 0^+} f(x). \)

q2720r1\\
\(\displaystyle -6 \)

q2720r2\\
\(\displaystyle -2 \)

q2720r3\\
\(\displaystyle 10 \)

q2720r4\\
\(\displaystyle \text{Does not exist} \)

q2720\\
\(\displaystyle \text{Let } f(x) = \frac{x^2 - 9}{x+3} \text{ for } x < -3, f(x) = 10 \text{ for } x = -3, f(x) = 3x+7 \text{ for } x > -3. \text{ Find } \lim_{x \rightarrow -3} f(x). \)

q2725r1\\
\(\displaystyle -1 \)

q2725r2\\
\(\displaystyle 0 \)

q2725r3\\
\(\displaystyle 1 \)

q2725r4\\
\(\displaystyle \text{Does not exist} \)

q2725\\
\(\displaystyle \text{Evaluate } \lim_{x \rightarrow 4} \frac{x+4}{x^2-16} \)

q2726r1\\
\(\displaystyle -1 \)

q2726r2\\
\(\displaystyle 0 \)

q2726r3\\
\(\displaystyle 1 \)

q2726r4\\
\(\displaystyle \text{Does not exist} \)

q2726\\
\(\displaystyle \text{Evaluate } \lim_{x \rightarrow 0^-} \frac{x}{|x|} \)

q2727r1\\
\(\displaystyle -1 \)

q2727r2\\
\(\displaystyle 0 \)

q2727r3\\
\(\displaystyle 1 \)

q2727r4\\
\(\displaystyle \text{Does not exist} \)

q2727\\
\(\displaystyle \text{Evaluate } \lim_{x \rightarrow \infty} \frac{1}{3x-7} \)

q2728r1\\
\(\displaystyle -6 \)

q2728r2\\
\(\displaystyle -2 \)

q2728r3\\
\(\displaystyle 10 \)

q2728r4\\
\(\displaystyle \text{Does not exist} \)

q2728\\
\(\displaystyle \text{Let } f(x) = \frac{x^2 - 9}{x+3} \text{ for } x < -3, f(x) = 10 \text{ for } x = -3, f(x) = 3x+7 \text{ for } x > -3. \text{ Find } \lim_{x \rightarrow -3^-} f(x). \)

q2729r1\\
\(\displaystyle -6 \)

q2729r2\\
\(\displaystyle -2 \)

q2729r3\\
\(\displaystyle 10 \)

q2729r4\\
\(\displaystyle \text{Does not exist} \)

q2729\\
\(\displaystyle \text{Let } f(x) = \frac{x^2 - 9}{x+3} \text{ for } x < -3, f(x) = 10 \text{ for } x = -3, f(x) = 3x+7 \text{ for } x > -3. \text{ Find } \lim_{x \rightarrow -3^+} f(x). \)

q2730r1\\
\(\displaystyle 0 \)

q2730r2\\
\(\displaystyle 2 \)

q2730r3\\
\(\displaystyle 4 \)

q2730r4\\
\(\displaystyle \text{Does not exist} \)

q2730\\
\(\displaystyle \text{Let } f(x) = 2x \text{ for } x < 1, f(x) = 4 \text{ for } x = 1, f(x) = x^2 + 1 \text{ for } x > 1. \text{ Find } \lim_{x \rightarrow 1} f(x). \)

q2735r1\\
\(\displaystyle -1 \)

q2735r2\\
\(\displaystyle 0 \)

q2735r3\\
\(\displaystyle 1 \)

q2735r4\\
\(\displaystyle \text{Does not exist} \)

q2735\\
\(\displaystyle \text{Evaluate } \lim_{x \rightarrow 0} \frac{|x|}{x} \)

q2736r1\\
\(\displaystyle -1 \)

q2736r2\\
\(\displaystyle 0 \)

q2736r3\\
\(\displaystyle 1 \)

q2736r4\\
\(\displaystyle \text{Does not exist} \)

q2736\\
\(\displaystyle \text{Evaluate } \lim_{x \rightarrow 2^+} \frac{\sqrt{x-2}}{3} \)

q2737r1\\
\(\displaystyle -1 \)

q2737r2\\
\(\displaystyle 0 \)

q2737r3\\
\(\displaystyle 1 \)

q2737r4\\
\(\displaystyle \text{Does not exist} \)

q2737\\
\(\displaystyle \text{Evaluate } \lim_{x \rightarrow \infty} \cos \left ( \frac{1}{x} \right ) \)

q2738r1\\
\(\displaystyle 0 \)

q2738r2\\
\(\displaystyle 2 \)

q2738r3\\
\(\displaystyle 4 \)

q2738r4\\
\(\displaystyle \text{Does not exist} \)

q2738\\
\(\displaystyle \text{Let } f(x) = 2x \text{ for } x < 1, f(x) = 4 \text{ for } x = 1, f(x) = x^2 + 1 \text{ for } x > 1. \text{ Find } \lim_{x \rightarrow 1^-} f(x). \)

q2739r1\\
\(\displaystyle 0 \)

q2739r2\\
\(\displaystyle 2 \)

q2739r3\\
\(\displaystyle 4 \)

q2739r4\\
\(\displaystyle \text{Does not exist} \)

q2739\\
\(\displaystyle \text{Let } f(x) = 2x \text{ for } x < 1, f(x) = 4 \text{ for } x = 1, f(x) = x^2 + 1 \text{ for } x > 1. \text{ Find } \lim_{x \rightarrow 1^+} f(x). \)

q3211r1\\
\(\displaystyle 2\pi/9 \)

q3211r2\\
\(\displaystyle 4\pi/9 \)

q3211r3\\
\(\displaystyle 1\pi/8 \)

q3211r4\\
\(\displaystyle 3\pi/10 \)

q3211\\
\(\displaystyle \text{Express } 40^{\circ} \text{ in radians}. \)

q3212r1\\
\(\displaystyle 210^{\circ} \)

q3212r2\\
\(\displaystyle 315^{\circ} \)

q3212r3\\
\(\displaystyle 330^{\circ} \)

q3212r4\\
\(\displaystyle 290^{\circ} \)

q3212\\
\(\displaystyle \text{Express } 7\pi/4 \text{ radians in degrees.} \)

q3213r1\\
\(\displaystyle 2 \)

q3213r2\\
\(\displaystyle -1 \)

q3213r3\\
\(\displaystyle 1 \)

q3213r4\\
\(\displaystyle 0 \)

q3213\\
\(\displaystyle \text{Evaluate } 2\log 3 + \log 4 - \log 36. \)

q3214r1\\
\(\displaystyle x^4/4 \)

q3214r2\\
\(\displaystyle -x^2 \)

q3214r3\\
\(\displaystyle 2-x^2/2 \)

q3214r4\\
\(\displaystyle 4/x^2 \)

q3214\\
\(\displaystyle \text{Evaluate } 2\ln (e^{-x^2/2}) \)

q3215r1\\
\(\displaystyle -\infty \)

q3215r2\\
\(\displaystyle \infty \)

q3215r3\\
\(\displaystyle 0 \)

q3215r4\\
\(\displaystyle 1 \)

q3215\\
\(\displaystyle \text{Evaluate } \lim_{x \rightarrow \infty} \ln \left ( \frac{x}{x+1} \right ) \)

q3216r1\\
\(\displaystyle \pi/6, 11\pi/6 \)

q3216r2\\
\(\displaystyle \pi/3, 5\pi/3 \)

q3216r3\\
\(\displaystyle 7\pi/6, 11\pi/6 \)

q3216r4\\
\(\displaystyle \pi/6, 7\pi/6 \)

q3216\\
\(\displaystyle \text{Find all solutions of } \sec q = \frac{2}{3^{1/2}} \text{ for } 0 \leq q < 2\pi. \)

q3217r1\\
\(\displaystyle 1 \)

q3217r2\\
\(\displaystyle 4 \)

q3217r3\\
\(\displaystyle -2 \)

q3217r4\\
\(\displaystyle 3 \)

q3217\\
\(\displaystyle \text{Solve } \log(x^2 - 3) = \log(2x). \)

q3218r1\\
\(\displaystyle -0.037 \)

q3218r2\\
\(\displaystyle 1.10 \)

q3218r3\\
\(\displaystyle -1.03 \)

q3218r4\\
\(\displaystyle -1.33 \)

q3218\\
\(\displaystyle \text{Solve } e^{-x^3} = 3. \text{ Approximate your answer to the nearest hundreth.} \)

q3221r1\\
\(\displaystyle 9\pi/8 \)

q3221r2\\
\(\displaystyle 7\pi/3 \)

q3221r3\\
\(\displaystyle 7\pi/6 \)

q3221r4\\
\(\displaystyle 6\pi/5 \)

q3221\\
\(\displaystyle \text{Express } 210^{\circ} \text{ in radians.} \)

q3222r1\\
\(\displaystyle -270^{\circ} \)

q3222r2\\
\(\displaystyle -210^{\circ} \)

q3222r3\\
\(\displaystyle -120^{\circ} \)

q3222r4\\
\(\displaystyle -240^{\circ} \)

q3222\\
\(\displaystyle \text{Express } -4\pi/3 \text{ radians in degrees.} \)

q3223r1\\
\(\displaystyle -1 \)

q3223r2\\
\(\displaystyle 0 \)

q3223r3\\
\(\displaystyle 1/4 \)

q3223r4\\
\(\displaystyle 2 \)

q3223\\
\(\displaystyle \text{Evaluate } 2\log_4 3^{1/2} - 4\log_4 2^{1/2} - \log_4 3 \)

q3224r1\\
\(\displaystyle xe^3 \)

q3224r2\\
\(\displaystyle x^3 \)

q3224r3\\
\(\displaystyle 3e^x \)

q3224r4\\
\(\displaystyle 3x \)

q3224\\
\(\displaystyle \text{Evaluate } e^{3 \ln x} \)

q3225r1\\
\(\displaystyle 0 \)

q3225r2\\
\(\displaystyle \infty \)

q3225r3\\
\(\displaystyle -\infty \)

q3225r4\\
\(\displaystyle 1 \)

q3225\\
\(\displaystyle \text{Evaluate } \lim_{x \rightarrow 0} \frac{1 - e^{-x}}{e^x} \)

q3226r1\\
\(\displaystyle \pi/6, 7\pi/6 \)

q3226r2\\
\(\displaystyle \pi/3, 4\pi/3 \)

q3226r3\\
\(\displaystyle \pi/3, 5\pi/3 \)

q3226r4\\
\(\displaystyle 2\pi/3, 4\pi/3 \)

q3226\\
\(\displaystyle \text{Find all solutions of } \cot q = \frac{3^{1/2}}{3} \text{ for } 0 \leq q < 2\pi. \)

q3227r1\\
\(\displaystyle 6 \)

q3227r2\\
\(\displaystyle -4 \)

q3227r3\\
\(\displaystyle 3 \)

q3227r4\\
\(\displaystyle -1 \)

q3227\\
\(\displaystyle \text{Solve } \log(x^2 - 4) - \log(x+2) = 0 \)

q3228r1\\
\(\displaystyle -0.34 \)

q3228r2\\
\(\displaystyle 0.83 \)

q3228r3\\
\(\displaystyle 1.41 \)

q3228r4\\
\(\displaystyle \text{No real solutions} \)

q3228\\
\(\displaystyle \text{Solve } e^{-x^2} + 2 = 0. \text{ Approximate your answer to the nearest hundreth.} \)

q3231r1\\
\(\displaystyle -2\pi/5 \)

q3231r2\\
\(\displaystyle -5\pi/12 \)

q3231r3\\
\(\displaystyle -3\pi/7 \)

q3231r4\\
\(\displaystyle -4\pi/9 \)

q3231\\
\(\displaystyle \text{Express } -80^{\circ} \text{ in radians.} \)

q3232r1\\
\(\displaystyle 15^{\circ} \)

q3232r2\\
\(\displaystyle 20^{\circ} \)

q3232r3\\
\(\displaystyle 10^{\circ} \)

q3232r4\\
\(\displaystyle 12^{\circ} \)

q3232\\
\(\displaystyle \text{Express } \pi/12 \text{ radians in degrees.} \)

q3233r1\\
\(\displaystyle -1 \)

q3233r2\\
\(\displaystyle 1 \)

q3233r3\\
\(\displaystyle 2 \)

q3233r4\\
\(\displaystyle 25 \)

q3233\\
\(\displaystyle \text{Evaluate } \log_5 \left ( \frac{125}{9} \right ) + 2\log_5 3 - \frac{\log_5 25}{2}. \)

q3234r1\\
\(\displaystyle -2x^{1/2} \)

q3234r2\\
\(\displaystyle 1/x \)

q3234r3\\
\(\displaystyle -2 + x^{1/2} \)

q3234r4\\
\(\displaystyle 2x \)

q3234\\
\(\displaystyle \text{Evaluate } -2\ln(e^{x^{1/2}}). \)

q3235r1\\
\(\displaystyle -3 \)

q3235r2\\
\(\displaystyle \infty \)

q3235r3\\
\(\displaystyle 3 \)

q3235r4\\
\(\displaystyle -\infty \)

q3235\\
\(\displaystyle \text{Evaluate } \lim_{x \rightarrow 0} -3\ln|x|. \)

q3236r1\\
\(\displaystyle 2\pi/3, 4\pi/3 \)

q3236r2\\
\(\displaystyle 2\pi/3, 5\pi/3 \)

q3236r3\\
\(\displaystyle 4\pi/3, 5\pi/3 \)

q3236r4\\
\(\displaystyle 7\pi/6, 10\pi/6 \)

q3236\\
\(\displaystyle \text{Find all solutions of } \csc q = \frac{-2}{3^{1/2}} \text{ for } 0 \leq q < 2\pi. \)

q3237r1\\
\(\displaystyle -1/4 \)

q3237r2\\
\(\displaystyle -1/2 \)

q3237r3\\
\(\displaystyle -1 \)

q3237r4\\
\(\displaystyle 0 \)

q3237\\
\(\displaystyle \text{Solve } 2\log(2x+1) = 0. \)

q3238r1\\
\(\displaystyle -0.14 \)

q3238r2\\
\(\displaystyle 0.14 \)

q3238r3\\
\(\displaystyle 0 \)

q3238r4\\
\(\displaystyle -3.47 \)

q3238\\
\(\displaystyle \text{Solve } 2e^{-5x} = 1. \text{ Approximate your answer to the nearest hundreth.} \)

q3511r1\\
\(\displaystyle 6(7x^3 - 2x^2 + 1)^5 \)

q3511r2\\
\(\displaystyle 6(21x^2 - 4x)(7x^3 - 2x^2 + 1)^5 \)

q3511r3\\
\(\displaystyle (21x^2 - 4x)(7x^3 - 2x^2 + 1)^5 \)

q3511r4\\
\(\displaystyle (21x^2 - 4x)^5 \)

q3511\\
\(\displaystyle (7x^3 - 2x^2 + 1)^6 \)

q3512r1\\
\(\displaystyle \frac{1}{3x^2 - 1} \)

q3512r2\\
\(\displaystyle \frac{1}{6x} \)

q3512r3\\
\(\displaystyle \frac{6x-1}{3x^2-1} \)

q3512r4\\
\(\displaystyle \frac{6x}{3x^2 - 1} \)

q3512\\
\(\displaystyle \ln(3x^2 - 1) \)

q3513r1\\
\(\displaystyle (8x-1)(4x^2 - x + 1)^{1/2} \)

q3513r2\\
\(\displaystyle \frac{8x - 1}{(4x^2 - x + 1)^{1/2}} \)

q3513r3\\
\(\displaystyle \frac{8x - 1}{2(4x^2 - x + 1)^{1/2}} \)

q3513r4\\
\(\displaystyle \frac{1}{2}(8x - 1)(4x^2 - x + 1)^{1/2} \)

q3513\\
\(\displaystyle (4x^2 - x + 1)^{1/2} \)

q3514r1\\
\(\displaystyle -10\sin(5x + 1)\cos(5x + 1) \)

q3514r2\\
\(\displaystyle -2\cos(5x + 1)\sin(5x + 1) \)

q3514r3\\
\(\displaystyle 10\cos(5x + 1) \)

q3514r4\\
\(\displaystyle 10\sin(5x + 1) \)

q3514\\
\(\displaystyle \cos^2 (5x + 1) \)

q3515r1\\
\(\displaystyle \frac{1}{4(2x - 3)} \)

q3515r2\\
\(\displaystyle \frac{-4}{(2x - 3)^3} \)

q3515r3\\
\(\displaystyle \frac{-2}{(2x - 3)^3} \)

q3515r4\\
\(\displaystyle \frac{2}{(2x -3)^3} \)

q3515\\
\(\displaystyle \frac{1}{(2x - 3)^2} \)

q3521r1\\
\(\displaystyle 3(2x^4 - 5x)^2 \)

q3521r2\\
\(\displaystyle 3(8x^3 - 5)(2x^4 - 5x)^2 \)

q3521r3\\
\(\displaystyle (8x^3 - 5)(2x^4 - 5x)^2 \)

q3521r4\\
\(\displaystyle (8x^3 - 5)^2 \)

q3521\\
\(\displaystyle (2x^4 - 5x)^3 \)

q3522r1\\
\(\displaystyle 2e^{(4x + 1)^2} \)

q3522r2\\
\(\displaystyle e^{8(4x + 1)} \)

q3522r3\\
\(\displaystyle 8(4x + 1)e^{(4x + 1)^2} \)

q3522r4\\
\(\displaystyle 2(4x + 1)e^{(4x + 1)^2} \)

q3522\\
\(\displaystyle e^{(4x + 1)^2} \)

q3523r1\\
\(\displaystyle \frac{2x}{3(x^2 + 1)^{2/3}} \)

q3523r2\\
\(\displaystyle \frac{1}{3(x^2 + 1)^{2/3}} \)

q3523r3\\
\(\displaystyle \frac{2x}{(x^2 + 1)^{2/3}} \)

q3523r4\\
\(\displaystyle 2x(x^2 + 1)^{1/3} \)

q3523\\
\(\displaystyle (x^2 + 1)^{1/3} \)

q3524r1\\
\(\displaystyle \frac{\cos[(2x)^{1/2}]}{(2(2x)^{1/2}} \)

q3524r2\\
\(\displaystyle \frac{\cos[(2x)^{1/2}]}{(2x)^{1/2}} \)

q3524r3\\
\(\displaystyle \cos[(2x)^{1/2}] \)

q3524r4\\
\(\displaystyle \frac{\cos[(2x)^{1/2}]}{2} \)

q3524\\
\(\displaystyle \sin[(2x)^{1/2}] \)

q3525r1\\
\(\displaystyle \frac{6x}{(3x^2 - 1)^2} \)

q3525r2\\
\(\displaystyle \frac{1}{6x} \)

q3525r3\\
\(\displaystyle \frac{1}{6x^2} \)

q3525r4\\
\(\displaystyle \frac{-6x}{(3x^2 - 1)^2} \)

q3525\\
\(\displaystyle \frac{1}{3x^2 - 1} \)

q3531r1\\
\(\displaystyle (12x^2 - 12x + 5)(4x^3 - 6x^2 + 5x + 3)^3 \)

q3531r2\\
\(\displaystyle (12x^2 - 12x + 5)^3 \)

q3531r3\\
\(\displaystyle 4(12x^2 - 12x + 5)(4x^3 - 6x^2 + 5x + 3)^3 \)

q3531r4\\
\(\displaystyle 4(4x^3 - 6x^2 + 5x + 3)^3 \)

q3531\\
\(\displaystyle (4x^3 - 6x^2 + 5x + 3)^4 \)

q3532r1\\
\(\displaystyle \frac{1}{(2x + 1)^{1/2}} \)

q3532r2\\
\(\displaystyle \frac{1}{2x + 1} \)

q3532r3\\
\(\displaystyle \frac{2}{(2x + 1)^{1/2}} \)

q3532r4\\
\(\displaystyle \frac{1}{(2(2x + 1)^{1/2}} \)

q3532\\
\(\displaystyle \ln[(2x + 1)^{1/2}] \)

q3533r1\\
\(\displaystyle \frac{3x^2}{4(x^3 - 2)^{3/4}} \)

q3533r2\\
\(\displaystyle 3x^2(x^3 - 2)^{5/4} \)

q3533r3\\
\(\displaystyle 3x^2(x^3 - 2)^{1/4} \)

q3533r4\\
\(\displaystyle \frac{3x^2}{(x^3 - 2)^{3/4}} \)

q3533\\
\(\displaystyle (x^3 - 2)^{1/4} \)

q3534r1\\
\(\displaystyle \frac{\sec^2(1/x)}{x^2} \)

q3534r2\\
\(\displaystyle \sec^2(1/x) \)

q3534r3\\
\(\displaystyle \frac{-\sec^2(1/x)}{x^2} \)

q3534r4\\
\(\displaystyle \sec^2(-1/x^2) \)

q3534\\
\(\displaystyle \tan(1/x) \)

q3535r1\\
\(\displaystyle \frac{5}{(5x - 2)^4} \)

q3535r2\\
\(\displaystyle \frac{-3}{(5x - 2)^4} \)

q3535r3\\
\(\displaystyle \frac{3}{(5x - 2)^2} \)

q3535r4\\
\(\displaystyle \frac{-15}{(5x - 2)^4} \)

q3535\\
\(\displaystyle \frac{1}{(5x - 2)^3} \)

q4211r1\\
\(\displaystyle \cos(2x)\cos(x^2+1) - \sin(2x)\sin(x^2+1) \)

q4211r2\\
\(\displaystyle 2\cos(2x)\cos(x^2+1) - 2x\sin(2x)\sin(x^2+1) \)

q4211r3\\
\(\displaystyle -4x\cos(2x)\sin(x^2+1) \)

q4211r4\\
\(\displaystyle 2\sin(2x)\sin(x^2+1) - 2x\cos(2x)\cos(x^2+1) \)

q4211\\
\(\displaystyle \text{Differentiate } f(x) = \sin(2x) \cos(x^2+1) \)

q4212r1\\
\(\displaystyle e^{x+1}( \ln(x) + x^{-1}) \)

q4212r2\\
\(\displaystyle e^{x+1} x^{-1} \)

q4212r3\\
\(\displaystyle e^{x+1} [ (x+1) \ln(x) + x^{-1}] \)

q4212r4\\
\(\displaystyle -e^{x+1} x^{-1} \)

q4212\\
\(\displaystyle \text{Differentiate } f(x) = e^{x+1} \ln x. \)

q4213r1\\
\(\displaystyle 2x^{-1} \sec^2(x) \)

q4213r2\\
\(\displaystyle \sec^2(x)\ln(x^2) + x^{-2}\tan(x) \)

q4213r3\\
\(\displaystyle x^{-2} \sec^2(x) \)

q4213r4\\
\(\displaystyle \sec^2(x)\ln(x^2) + 2x^{-1}\tan(x) \)

q4213\\
\(\displaystyle \text{Differentiate } f(x) = \tan(x) \ln(x^2). \)

q4214r1\\
\(\displaystyle \pi^3/2 \)

q4214r2\\
\(\displaystyle \pi^4/16 \)

q4214r3\\
\(\displaystyle -\pi^3/2 \)

q4214r4\\
\(\displaystyle -\pi^4/16 \)

q4214\\
\(\displaystyle \text{If } f(x) = x^4 \cos(x), \text{ what is } f'(\pi/2)? \)

q4215r1\\
\(\displaystyle 2 \)

q4215r2\\
\(\displaystyle 10 \)

q4215r3\\
\(\displaystyle 12 \)

q4215r4\\
\(\displaystyle 18 \)

q4215\\
\(\displaystyle \text{If } f(2) = 4 \text{ and } f'(2) = 1, \text{ find } g'(2) \text{ if } g(x) = 2xf(x). \)

q4221r1\\
\(\displaystyle 6x + 3\cos(3x) \)

q4221r2\\
\(\displaystyle 18x\cos(3x) + 3x^2\sin(3x) \)

q4221r3\\
\(\displaystyle 6x\sin(3x) + 9x^2\cos(3x) \)

q4221r4\\
\(\displaystyle 18x\cos(3x) \)

q4221\\
\(\displaystyle \text{Differentiate } f(x) = 3x^2 \sin(3x). \)

q4222r1\\
\(\displaystyle abe^{abx} (ax^b + x^{b-1} + abx + 1) \)

q4222r2\\
\(\displaystyle a^2b^2e^{abx}(x^{b-1} + 1) \)

q4222r3\\
\(\displaystyle a^2 b^2 e^{abx} (x^b + x^{b-1} + x + 1) \)

q4222r4\\
\(\displaystyle abe^{abx} (ax^b + abx)(x^{b-1} + 1) \)

q4222\\
\(\displaystyle \text{Differentiate } f(x) = e^{abx} (ax^b + abx). \)

q4223r1\\
\(\displaystyle 2xe^{x^2} \)

q4223r2\\
\(\displaystyle 4x^2 e^{x^2} \)

q4223r3\\
\(\displaystyle 2x e^{x^2} (1+x^2) \)

q4223r4\\
\(\displaystyle 4x^2 e^{x^2} (1+x^2) \)

q4223\\
\(\displaystyle \text{Differentiate } f(x) = x^2 e^{x^2}. \)

q4224r1\\
\(\displaystyle 9 \)

q4224r2\\
\(\displaystyle 3 \)

q4224r3\\
\(\displaystyle 6 \)

q4224r4\\
\(\displaystyle 1 \)

q4224\\
\(\displaystyle \text{If } f(x) = (3x^3 + 3x) e^{3x}, \text{ what is } f'(0)? \)

q4225r1\\
\(\displaystyle -4 \)

q4225r2\\
\(\displaystyle -2 \)

q4225r3\\
\(\displaystyle 2 \)

q4225r4\\
\(\displaystyle 4 \)

q4225\\
\(\displaystyle \text{If } f(0) = -2 \text{ and } f'(0) = 4, \text{ find } g'(0) \text { if } g(x) = \cos(x + \pi/2) f(x). \)

q4231r1\\
\(\displaystyle -2e^{-x^2} \)

q4231r2\\
\(\displaystyle (x^{-1} - 2x\ln(x)) e^{-x^2} \)

q4231r3\\
\(\displaystyle (x^{-1} - 2x) e^{-x^2} \ln(x) \)

q4231r4\\
\(\displaystyle -2e^{-x^2} \ln(x) \)

q4231\\
\(\displaystyle \text{Differentiate } f(x) = e^{-x^2} \ln(x). \)

q4232r1\\
\(\displaystyle e^x (\cos(x) + \sin(x)) \)

q4232r2\\
\(\displaystyle e^x \sin(x) \)

q4232r3\\
\(\displaystyle -e^x \sin(x) \)

q4232r4\\
\(\displaystyle e^x (\cos(x) - \sin(x)) \)

q4232\\
\(\displaystyle \text{Differentiate } f(x) = e^x \cos(x). \)

q4233r1\\
\(\displaystyle \cos^2(x) - \sin^2(x) \)

q4233r2\\
\(\displaystyle 1 \)

q4233r3\\
\(\displaystyle \sin^2(x) - \cos^2(x) \)

q4233r4\\
\(\displaystyle \sin(x) - \cos(x) \)

q4233\\
\(\displaystyle \text{Differentiate } f(x) = \sin(x) \cos(x). \)

q4234r1\\
\(\displaystyle 0 \)

q4234r2\\
\(\displaystyle 7\pi/2 \)

q4234r3\\
\(\displaystyle 6\pi \)

q4234r4\\
\(\displaystyle 12 \)

q4234\\
\(\displaystyle \text{If } f(x) = (x^4 + 2x^2 + 4x) \sin(\pi x / 2), \text{ what is } f'(1)? \)

q4235r1\\
\(\displaystyle -10 \)

q4235r2\\
\(\displaystyle -5 \)

q4235r3\\
\(\displaystyle 0 \)

q4235r4\\
\(\displaystyle 5 \)

q4235\\
\(\displaystyle \text{If } f(1) = 5 \text{ and } f'(1) = -5, \text{ find } g'(1) \text{ if } g(x) = e^{2x-2} f(x). \)

q4310\\
\(\displaystyle \text{If } mx = nx^2 \text{ then } m = nx. \)

q4311\\
\(\displaystyle (z-3)^2 = z^2 + 9 \)

q4312\\
\(\displaystyle \frac{1}{x} + \frac{1}{x^2} = \frac{x+1}{x^2} \)

q4313\\
\(\displaystyle \lim_{x \rightarrow 2} \frac{x-2}{x^2-4} = \frac{1}{2} \)

q4314\\
\(\displaystyle \frac{d}{dx} [\cos(x^3)] = -3x^2\sin(3x^2) \)

q4315\\
\(\displaystyle \frac{d}{dx} [\ln(5^{1/2})] = 0 \)

q4316\\
\(\displaystyle \int e^x \, dx = e^x \)

q4317\\
\(\displaystyle \text{If } x^{3/2} = x^{1/2}, \text{ then } x = 1. \)

q4318\\
\(\displaystyle (x^2+16)^{1/2} = x+4 \)

q4319\\
\(\displaystyle \lim_{x \rightarrow 1} \ln \frac{x}{x} = 0 \)

q4320\\
\(\displaystyle \frac{d}{dx} \left [ \frac{x}{x+1} \right ] = \frac{1}{(x+1)^2} \)

q4321\\
\(\displaystyle \frac{1}{z} - z = \frac{1-z^2}{z} \)

q4322\\
\(\displaystyle \frac{\tan(6x)}{3} = \tan(2x) \)

q4323\\
\(\displaystyle \frac{d}{dx} [ \cos(2x)] = 2\sin(2x) \)

q4324\\
\(\displaystyle \int \frac{1}{4x}\,dx = \frac{1}{4} \ln |x| + C \)

q4325\\
\(\displaystyle 4x^2 - 9z^2 = (2x+3z)(2x-3z) \)

q4326\\
\(\displaystyle \frac{d}{dx} [3^{-x}] = -x3^{-x-1} \)

q4327\\
\(\displaystyle \frac{d}{dx}[\sin(2x)] = \frac{1}{2} \cos(2x) \)

q4328\\
\(\displaystyle \text{If } y < z, \text{ then } y/3 < z/3. \)

q4329\\
\(\displaystyle \int \frac{x^4}{x^3} \, dx = \frac{x^2}{2} + C \)

q4330\\
\(\displaystyle \frac{d}{dx} \left [ \frac{x}{2x+1} \right ] = \frac{1}{2} \)

q4331\\
\(\displaystyle (x+x^{1/2})^2 = x^2 + x \)

q4332\\
\(\displaystyle (m^2 - n^2)^{1/2} = m - n \)

q4333\\
\(\displaystyle \text{If } x^2 + 1 = x^3 + x, \text{ then } x = 1. \)

q4334\\
\(\displaystyle \frac{d}{dx} [\cos(2x)] = 2\sin(2x) \)

q4335\\
\(\displaystyle \int \frac{1}{x^{1/x}} \, dx = \ln | x^{1/2} | + C \)

q4336\\
\(\displaystyle (x^2+1)^2 = x^4 + 2x^2 + 1 \)

q4337\\
\(\displaystyle \frac{1}{y} + \frac{1}{2y} = \frac{1}{3y} \)

q4338\\
\(\displaystyle \text{If } x^2 > 3x, \text{ then } x > 3. \)

q4339\\
\(\displaystyle \frac{d}{dx} [3^{-x}] = -3^{-x} \)

q4340\\
\(\displaystyle \frac{d}{dx} [\cos(x^3)] = -3x^2\sin(x^3) \)

q4341\\
\(\displaystyle \int x^{1/2}\,dx = \frac{1}{2x^{1/2}} + C \)

q4342\\
\(\displaystyle \frac{q}{p} - \frac{p}{q} = \frac{q^2 - p^2}{pq} \)

q4343\\
\(\displaystyle \cos(\pi x) / \pi = \cos x \)

q4344\\
\(\displaystyle \frac{d}{dx} [\sin(5x^2)] = \cos(5x^2) + 10x \)

q4345\\
\(\displaystyle \int 3\,dx = 3x \)

q4346\\
\(\displaystyle (2z - z^{-1})^2 = 4z^2 - 4 + z^{-2} \)

q4347\\
\(\displaystyle \sin^2(\pi x) - \cos^2(\pi x) = 1 \)

q4348\\
\(\displaystyle \lim_{x \rightarrow 0} \frac{x^3}{x+1} = 0 \)

q4349\\
\(\displaystyle \frac{d}{dx} [ \sin(\pi/2) ] = \pi/2 \cos(\pi/2) \)

q4350\\
\(\displaystyle \lim_{x \rightarrow \infty} \frac{x^{1/2}}{x^{10}-3} = 0 \)

q4351\\
\(\displaystyle [(x^2+3)^{1/2}]^2 = x^2 + 3 \)

q4352\\
\(\displaystyle \frac{d}{dx}[\cos(x^3)] = -\sin(3x^2) \)

q4353\\
\(\displaystyle \frac{d}{dx} e^{-3x} = -3e^{-3x} \)

q4354\\
\(\displaystyle \int x^2\,dx = \frac{x^3}{3} \)

q4355\\
\(\displaystyle \int \ln |x| \, dx = \frac{1}{x} + C \)

q4356\\
\(\displaystyle (e^x)^0 = 1 \)

q4357\\
\(\displaystyle \int \frac{1}{2x}\,dx = \frac{1}{2} + C \)

q4358\\
\(\displaystyle \frac{\sin(3x)}{\cos(3x)} = \tan(3x) \)

q4359\\
\(\displaystyle \frac{1}{x+1/y} = \frac{y}{x} \)

q4412r1\\
\(\displaystyle -2 \)

q4412r2\\
\(\displaystyle -1 \)

q4412r3\\
\(\displaystyle 1 \)

q4412r4\\
\(\displaystyle 2 \)

q4412\\
\(\displaystyle \text{Find the critical points of } f(x). \)

q4413r1\\
\(\displaystyle (-2, -1/2) \)

q4413r2\\
\(\displaystyle (-1, 5) \)

q4413r3\\
\(\displaystyle (2, 17/2) \)

q4413r4\\
\(\displaystyle \text{There are none.} \)

q4413\\
\(\displaystyle \text{Find the location of relative maxima of } f(x). \)

q4414r1\\
\(\displaystyle (-2, -1/2) \)

q4414r2\\
\(\displaystyle (1, -5) \)

q4414r3\\
\(\displaystyle (2, -17/2) \)

q4414r4\\
\(\displaystyle \text{There are none.} \)

q4414\\
\(\displaystyle \text{Find the location of relative minima of } f(x). \)

q4415r1\\
\(\displaystyle -5 \)

q4415r2\\
\(\displaystyle -1 \)

q4415r3\\
\(\displaystyle -1/2 \)

q4415r4\\
\(\displaystyle 5 \)

q4415\\
\(\displaystyle \text{What is the maximum value of } f(x) \text{ on the interval } [-3, 1]? \)

q4416r1\\
\(\displaystyle -21 \)

q4416r2\\
\(\displaystyle -17/2 \)

q4416r3\\
\(\displaystyle -5 \)

q4416r4\\
\(\displaystyle -3 \)

q4416\\
\(\displaystyle \text{What is the minimum value of } f(x) \text{ on the interval } [-3, 1]? \)

q4422r1\\
\(\displaystyle -1 \)

q4422r2\\
\(\displaystyle 0 \)

q4422r3\\
\(\displaystyle 1 \)

q4422r4\\
\(\displaystyle 2 \)

q4422\\
\(\displaystyle \text{Find all critical points of } f(x). \)

q4423r1\\
\(\displaystyle (-1, 0) \)

q4423r2\\
\(\displaystyle (0, 1/4) \)

q4423r3\\
\(\displaystyle (1, 4) \)

q4423r4\\
\(\displaystyle \text{There are none.} \)

q4423\\
\(\displaystyle \text{Find the location of relative maxima of } f(x). \)

q4424r1\\
\(\displaystyle (-1, 0) \)

q4424r2\\
\(\displaystyle (0, 1/4) \)

q4424r3\\
\(\displaystyle (1, 4) \)

q4424r4\\
\(\displaystyle \text{There are none.} \)

q4424\\
\(\displaystyle \text{Find the location of relative minima of } f(x). \)

q4425r1\\
\(\displaystyle 0 \)

q4425r2\\
\(\displaystyle 1/4 \)

q4425r3\\
\(\displaystyle 1 \)

q4425r4\\
\(\displaystyle 4 \)

q4425\\
\(\displaystyle \text{What is the maximum value of } f(x) \text{ on the interval } [-3, 1]? \)

q4426r1\\
\(\displaystyle 0 \)

q4426r2\\
\(\displaystyle 1/4 \)

q4426r3\\
\(\displaystyle 1 \)

q4426r4\\
\(\displaystyle 4 \)

q4426\\
\(\displaystyle \text{What is the minimum value of } f(x) \text{ on the interval } [-3, 1]? \)

q4432r1\\
\(\displaystyle -1/2 \)

q4432r2\\
\(\displaystyle 0 \)

q4432r3\\
\(\displaystyle 1/2 \)

q4432r4\\
\(\displaystyle 3/2 \)

q4432\\
\(\displaystyle \text{Find the critical points of } f(x). \)

q4433r1\\
\(\displaystyle (-1/2, 2/3) \)

q4433r2\\
\(\displaystyle (0, 1/3) \)

q4433r3\\
\(\displaystyle (1/2, 0) \)

q4433r4\\
\(\displaystyle \text{There are none.} \)

q4433\\
\(\displaystyle \text{Find the location of relative maxima of } f(x). \)

q4434r1\\
\(\displaystyle (-1/2, 2/3) \)

q4434r2\\
\(\displaystyle (0, 1/3) \)

q4434r3\\
\(\displaystyle (1/2, 0) \)

q4434r4\\
\(\displaystyle \text{There are none.} \)

q4434\\
\(\displaystyle \text{Find the location of relative minima of } f(x). \)

q4435r1\\
\(\displaystyle 0 \)

q4435r2\\
\(\displaystyle 1/3 \)

q4435r3\\
\(\displaystyle 2/3 \)

q4435r4\\
\(\displaystyle 1 \)

q4435\\
\(\displaystyle \text{What is the maximum value of } f(x) \text{ on the interval } [-3, 1]? \)

q4436r1\\
\(\displaystyle 0 \)

q4436r2\\
\(\displaystyle -98/3 \)

q4436r3\\
\(\displaystyle 2/3 \)

q4436r4\\
\(\displaystyle 1 \)

q4436\\
\(\displaystyle \text{What is the minimum value of } f(x) \text{ on the interval } [-3, 1]? \)

q4511r1\\
\(\displaystyle x < -2 \)

q4511r2\\
\(\displaystyle x > -2 \)

q4511r3\\
\(\displaystyle x > -2/3 \)

q4511r4\\
\(\displaystyle x < -2/3 \)

q4511\\
\(\displaystyle \text{Where is } f(x) \text{ concave up?} \)

q4512r1\\
\(\displaystyle (-2, 13) \)

q4512r2\\
\(\displaystyle (-2/3, 223/27) \)

q4512r3\\
\(\displaystyle (0, 5) \)

q4512r4\\
\(\displaystyle (2/3, 95/27) \)

q4512\\
\(\displaystyle \text{Find all inflection points of } f(x). \)

q4513r1\\
\(\displaystyle x = -2 \)

q4513r2\\
\(\displaystyle x = -2/3 \)

q4513r3\\
\(\displaystyle x = 2/3 \)

q4513r4\\
\(\displaystyle x = -3 \)

q4513\\
\(\displaystyle \text{At what value(s) of } x \text{ does } f(x) \text{ have an relative maximum?} \)

q4514r1\\
\(\displaystyle x = -2 \)

q4514r2\\
\(\displaystyle x = -2/3 \)

q4514r3\\
\(\displaystyle x = 2/3 \)

q4514r4\\
\(\displaystyle x = -3 \)

q4514\\
\(\displaystyle \text{At what value(s) of } x \text{ does } f(x) \text{ have a relative minimum?} \)

q4515r1\\
\(\displaystyle \text{A} \)

q4515r2\\
\(\displaystyle \text{B} \)

q4515r3\\
\(\displaystyle \text{C} \)

q4515r4\\
\(\displaystyle \text{D} \)

q4515\\
\(\displaystyle \text{Qualitatively, which plot below could be the graph of } y = f(x)? \)

q4521r1\\
\(\displaystyle x < 3^{1/2} \)

q4521r2\\
\(\displaystyle x < 0 \)

q4521r3\\
\(\displaystyle x > 0 \)

q4521r4\\
\(\displaystyle x > -3^{1/2} \)

q4521\\
\(\displaystyle \text{Where is } f(x) \text{ concave down?} \)

q4522r1\\
\(\displaystyle (1, 10) \)

q4522r2\\
\(\displaystyle (-3^{1/2}, 2 - 6(3)^{1/2}) \)

q4522r3\\
\(\displaystyle (3^{1/2}, 2 + 6(3)^{1/2}) \)

q4522r4\\
\(\displaystyle (0, 2) \)

q4522\\
\(\displaystyle \text{Find all inflection points of } f(x). \)

q4523r1\\
\(\displaystyle 3^{1/2} \)

q4523r2\\
\(\displaystyle -3^{1/2} \)

q4523r3\\
\(\displaystyle 0 \)

q4523r4\\
\(\displaystyle 1 \)

q4523\\
\(\displaystyle \text{At what value(s) of } f(x) \text{ does } f(x) \text{ have a relative maximum?} \)

q4524r1\\
\(\displaystyle 3^{1/2} \)

q4524r2\\
\(\displaystyle -3^{1/2} \)

q4524r3\\
\(\displaystyle 0 \)

q4524r4\\
\(\displaystyle 1 \)

q4524\\
\(\displaystyle \text{At what value(s) of } x \text{ does } f(x) \text{ have a relative minimum?} \)

q4525r1\\
\(\displaystyle \text{A} \)

q4525r2\\
\(\displaystyle \text{B} \)

q4525r3\\
\(\displaystyle \text{C} \)

q4525r4\\
\(\displaystyle \text{D} \)

q4525\\
\(\displaystyle \text{Qualitatively, which plot below could be the graph of } y = f(x)? \)

q4531r1\\
\(\displaystyle x > 2 \)

q4531r2\\
\(\displaystyle x > 5/2 \)

q4531r3\\
\(\displaystyle x > -5/2 \)

q4531r4\\
\(\displaystyle x < 5/2 \)

q4531\\
\(\displaystyle \text{Let } f(x) = 2x^3 - 15x^2 + 36x - 7. \text{Where is } f(x) \text{ concave up?} \)

q4532r1\\
\(\displaystyle (2, 21) \)

q4532r2\\
\(\displaystyle (3, 20) \)

q4532r3\\
\(\displaystyle (-5/2, -222) \)

q4532r4\\
\(\displaystyle (5/2, 41/2) \)

q4532\\
\(\displaystyle \text{Find all inflection points of } f(x). \)

q4533r1\\
\(\displaystyle x = 2 \)

q4533r2\\
\(\displaystyle x = 3 \)

q4533r3\\
\(\displaystyle x = -5/2 \)

q4533r4\\
\(\displaystyle x = 5/2 \)

q4533\\
\(\displaystyle \text{At what value(s) of } x \text{ does } f(x) \text{ have a relative maximum?} \)

q4534r1\\
\(\displaystyle x = 2 \)

q4534r2\\
\(\displaystyle x = 3 \)

q4534r3\\
\(\displaystyle x = -5/2 \)

q4534r4\\
\(\displaystyle x = 5/2 \)

q4534\\
\(\displaystyle \text{At what value(s) of } x \text{ does } f(x) \text{ have a relative minimum?} \)

q4537\\
\(\displaystyle \int \frac{x^{3/2}}{x^{1/2}}\,dx = \frac{x^2}{2} + C \)

q4538\\
\(\displaystyle \frac{1}{x} + \frac{1}{2} = \frac{1}{x+2} \)

q4539\\
\(\displaystyle \frac{d}{dx} [\sin(5x^2)] = 10x \cos(5x^2) \)

q4540\\
\(\displaystyle \frac{d}{dx} \left [ \frac{x}{x^2-2} \right ] = \frac{x^2+2}{(x^2-2)^2} \)

q4542\\
\(\displaystyle \text{If } (x^2+1)^{1/2} < (y^2 + 1)^{1/2} \text{ then } x^2 < y^2. \)

q4543\\
\(\displaystyle (9x^4 + 16x^4)^{1/2} = 5x^2 \)

q4544\\
\(\displaystyle \frac{d}{dx} [\cos(\pi x)] = -\pi \sin(\pi x) \)

q4545\\
\(\displaystyle \int x^5\,dx = 5x^4 + C \)

q4546\\
\(\displaystyle \int \frac{x^2-1}{x+1}\,dx = \frac{x^2}{2} - x + C \)

q4547\\
\(\displaystyle \frac{d}{dx} [3^{-4}] = 0 \)

q4548r1\\
\(\displaystyle (-\infty, -1) \)

q4548r2\\
\(\displaystyle (-2, 1) \)

q4548r3\\
\(\displaystyle (-1, 2) \)

q4548r4\\
\(\displaystyle (2, \infty) \)

q4548\\
\(\displaystyle \text{On which intervals is } f(x) \text{ increasing?} \)

q4549r1\\
\(\displaystyle (-\infty, -1) \)

q4549r2\\
\(\displaystyle (-1, \infty) \)

q4549r3\\
\(\displaystyle (1, \infty) \)

q4549r4\\
\(\displaystyle (-1, 1) \)

q4549\\
\(\displaystyle \text{On which intervals is } f(x) \text{ decreasing?} \)

q4550r1\\
\(\displaystyle (-\infty, -1/2) \)

q4550r2\\
\(\displaystyle (-1/2, 1/2) \)

q4550r3\\
\(\displaystyle (-1/2, \infty) \)

q4550r4\\
\(\displaystyle (1/2, \infty) \)

q4550\\
\(\displaystyle \text{On which intervals is } f(x) \text{ decreasing?} \)

q4551r1\\
\(\displaystyle A \)

q4551r2\\
\(\displaystyle B \)

q4551r3\\
\(\displaystyle C \)

q4551r4\\
\(\displaystyle D \)

q4551\\
\(\displaystyle \text{Which of the above could be the graph of f(x)?} \)

q4552r1\\
\(\displaystyle A \)

q4552r2\\
\(\displaystyle B \)

q4552r3\\
\(\displaystyle C \)

q4552r4\\
\(\displaystyle D \)

q4552\\
\(\displaystyle \text{Which of the above could be the graph of } f(x)? \)

q4553r1\\
\(\displaystyle A \)

q4553r2\\
\(\displaystyle B \)

q4553r3\\
\(\displaystyle C \)

q4553r4\\
\(\displaystyle D \)

q4553\\
\(\displaystyle \text{Which of the above could be the graph of } f(x)? \)

q4561r1\\
\(\displaystyle \frac{(x - x^3)^4(3x^2 - 1)}{3} \)

q4561r2\\
\(\displaystyle 3(x - x^3)^2(1 - 3x^2) \)

q4561r3\\
\(\displaystyle 3(x - x^3)^2 \)

q4561r4\\
\(\displaystyle 3 \left ( \frac{x^2}{2} - \frac{x^4}{4} \right ) ^2 \)

q4561\\
\(\displaystyle (x - x^3)^3 \)

q4562r1\\
\(\displaystyle 20(5x^4 + 10x^9)^{19} \)

q4562r2\\
\(\displaystyle 20(x^5 - x^{10})^{19} \)

q4562r3\\
\(\displaystyle 20(5x^4 - 10x^9)^{19} \)

q4562r4\\
\(\displaystyle 20(x^5 - x^{10})^{19} (5x^4 - 10x^9) \)

q4562\\
\(\displaystyle (x^5 - x^{10})^{20} \)

q4569r1\\
\(\displaystyle 4 \left ( \frac{1}{x} + \frac{1}{x^2} \right ) ^3 \)

q4569r2\\
\(\displaystyle \frac{1}{5} \left ( \frac{1}{x} + \frac{1}{x^2} \right ) ^5 \)

q4569r3\\
\(\displaystyle 4 \left ( \frac{1}{x} - \frac{1}{x^2} \right)^3 \left ( \frac{1}{x^2} + \frac{2}{x^3} \right ) \)

q4569r4\\
\(\displaystyle -4\left ( \frac{1}{x} + \frac{1}{x^2} \right )^3 \left ( \frac{1}{x^2} + \frac{2}{x^3} \right ) \)

q4569\\
\(\displaystyle \left ( \frac{1}{x} + \frac{1}{x^2} \right ) ^4 \)

q4570r1\\
\(\displaystyle 3 \cos 3x f'(\sin 3x) \)

q4570r2\\
\(\displaystyle 3 \sin 3x f'(x) \)

q4570r3\\
\(\displaystyle 3\cos 3x f'(x) \)

q4570r4\\
\(\displaystyle f'(3 \cos 3x) \)

q4570\\
\(\displaystyle f(\sin 3x) \)

q4571r1\\
\(\displaystyle 3[2(3x^2 + 3x^{-4}) - 2x]^2 \)

q4571r2\\
\(\displaystyle 3[(x^3 - x^{-3})^2 - x^2]^2 [2(x^3 - x^{-3})(3x^2+ 3x^{-4}) - 2x] \)

q4571r3\\
\(\displaystyle 3[(x^3 - x^{-3})^2 - x^2] \)

q4571r4\\
\(\displaystyle 3[(x^3 - x^{-3})^2 - x^2]^2[2(x^3 - x^{-3}) - 2x] \)

q4571\\
\(\displaystyle [(x^3 - x^{-3})^2 - x^2]^3 \)

q4572r1\\
\(\displaystyle 200x(x^2 - 1)^{99} \)

q4572r2\\
\(\displaystyle 100x(x^2 - 1)^{99} \)

q4572r3\\
\(\displaystyle \frac{(x^2 - 1)^{101}}{101} \)

q4572r4\\
\(\displaystyle 100x^2(x^2 - 1)^{99} \)

q4572\\
\(\displaystyle (x^2 - 1)^{100} \)

q4573r1\\
\(\displaystyle -3 \left ( x + \frac{1}{x} \right ) ^{-4} \left ( 1 + \frac{1}{x^2} \right ) \)

q4573r2\\
\(\displaystyle -3 \left ( x - \frac{1}{x} \right )^{-4} \left ( 1 + \frac{1}{x^2} \right ) \)

q4573r3\\
\(\displaystyle -3 \left ( x + \frac{1}{x} \right ) ^{-4} \left (1 - \frac{1}{x^2} \right ) \)

q4573r4\\
\(\displaystyle 3 \left ( x + \frac{1}{x} \right )^{-4} \left ( 1 - \frac{1}{x^2} \right ) \)

q4573\\
\(\displaystyle \left ( x + \frac{1}{x} \right ) ^{-3} \)

q4574r1\\
\(\displaystyle 18(2 - 3x) [1 - (2 + 3x)^2]^2 \)

q4574r2\\
\(\displaystyle 3[1 - (2 + 3x)^2]^2 \)

q4574r3\\
\(\displaystyle -18(2 + 3x)[1 - (2+3x)^2]^2 \)

q4574r4\\
\(\displaystyle 3[1 - 2(2 + 3x)]^2 \)

q4574\\
\(\displaystyle [1 - (2 + 3x)^2]^3 \)

q4577r1\\
\(\displaystyle 2 \tan x \)

q4577r2\\
\(\displaystyle 2x \sec^2 x^2 \)

q4577r3\\
\(\displaystyle \sec^2 x^2 \)

q4577r4\\
\(\displaystyle \tan 2x \)

q4577\\
\(\displaystyle \tan x^2 \)

q4578r1\\
\(\displaystyle 4\pi (x + \cot \pi x)^3 \)

q4578r2\\
\(\displaystyle 4 (x + \cot \pi x)^3 \)

q4578r3\\
\(\displaystyle 4(1 + \pi \csc^2 \pi x)^3 \)

q4578r4\\
\(\displaystyle 4(1 - \pi \csc^2 \pi x)(x + \cot \pi x)^3 \)

q4578\\
\(\displaystyle (x + \cot \pi x)^4 \)

q4579r1\\
\(\displaystyle \frac{1}{5} x^2(x^2 - 1)^5 + C \)

q4579r2\\
\(\displaystyle 8x (x^2 - 1)^3 + C \)

q4579r3\\
\(\displaystyle \frac{1}{5} (x^2 - 1)^5 + C \)

q4579r4\\
\(\displaystyle \frac{2}{5} x^2(x^2 - 1)^5 + C \)

q4579\\
\(\displaystyle \int 2x (x^2 - 1)^4 \, dx. \)

q4580r1\\
\(\displaystyle \frac{-1}{5(3+5x)} + C \)

q4580r2\\
\(\displaystyle \frac{-1}{3 + 5x} + C \)

q4580r3\\
\(\displaystyle \ln | 3 + 5x|^2 + C \)

q4580r4\\
\(\displaystyle \frac{-2}{(3+5x)^3} + C \)

q4580\\
\(\displaystyle \int \frac{1}{(3+5x)^2}\,dx. \)

q4581r1\\
\(\displaystyle 80 \)

q4581r2\\
\(\displaystyle 40 \)

q4581r3\\
\(\displaystyle 20 \)

q4581r4\\
\(\displaystyle 5 \)

q4581\\
\(\displaystyle \int_0^2 (x^2 - 1)(x^3 - 3x + 2)^3 \, dx \)

q4582r1\\
\(\displaystyle 2\pi \)

q4582r2\\
\(\displaystyle \frac{1}{4\pi} \)

q4582r3\\
\(\displaystyle \frac{1}{2\pi} \)

q4582r4\\
\(\displaystyle 0 \)

q4582\\
\(\displaystyle \int_0^{1/2} \cos^3(\pi x) \sin (\pi x) \, dx \)

q4583r1\\
\(\displaystyle \frac{|a|^3}{6} \)

q4583r2\\
\(\displaystyle \frac{|a|^3}{12} \)

q4583r3\\
\(\displaystyle 0 \)

q4583r4\\
\(\displaystyle \frac{|a|^3}{3} \)

q4583\\
\(\displaystyle \int_0^a y(a^2 - y^2)^{1/2}\,dy \)

q4584r1\\
\(\displaystyle \frac{\pi}{2} \)

q4584r2\\
\(\displaystyle \frac{3^{1/2} - 2^{1/2}}{\pi} \)

q4584r3\\
\(\displaystyle \frac{3^{1/2} - 1}{\pi} \)

q4584r4\\
\(\displaystyle \frac{1 - 3^{1/2}}{\pi} \)

q4584\\
\(\displaystyle \int_{1/4}^{1/3} \sec^2 \pi x \, dx \)

q4585r1\\
\(\displaystyle 31/2 \)

q4585r2\\
\(\displaystyle 125/2 \)

q4585r3\\
\(\displaystyle 80 \)

q4585r4\\
\(\displaystyle 23/4 \)

q4585\\
\(\displaystyle \int_0^1 5x (1 + x^2)^4 \, dx \)

q4586r1\\
\(\displaystyle \frac{1}{75} t^3 (5t^3 + 9)^5 + C \)

q4586r2\\
\(\displaystyle \frac{1}{96} t^3 (5t^3 + 9)^5 + C \)

q4586r3\\
\(\displaystyle \frac{1}{15} t^3 (5t^3 + 9)^5 + C \)

q4586r4\\
\(\displaystyle \frac{1}{75} (5t^3 + 9)^5 + C \)

q4586\\
\(\displaystyle \int t^2 (5t^3 + 9)^4\,dt \)

q4587r1\\
\(\displaystyle \frac{2}{3} (x+1)^{3/2} + 2x \)

q4587r2\\
\(\displaystyle (x+1)^{1/2} \left ( \frac{2}{3} x + 5 \right ) \)

q4587r3\\
\(\displaystyle \frac{1}{4} (x+3)(1 - x) \)

q4587r4\\
\(\displaystyle \frac{2}{3} (x+1)^{1/2} (x+37) \)

q4587\\
\(\displaystyle \int \frac{x+13}{(x+1)^{1/2}}\,dx \)

q4588r1\\
\(\displaystyle 1/2 \)

q4588r2\\
\(\displaystyle 1/4 \)

q4588r3\\
\(\displaystyle 0 \)

q4588r4\\
\(\displaystyle 1/8 \)

q4588\\
\(\displaystyle \int_{-1}^1 \frac{r}{(1+r^2)^4}\,dr \)

q4606r1\\
\(\displaystyle \left ( \frac{x}{2} - \frac{1}{4} \right ) e^{-2x} + C \)

q4606r2\\
\(\displaystyle \left ( \frac{x}{2} + \frac{1}{4} \right ) e^{-2x} + C \)

q4606r3\\
\(\displaystyle \left ( \frac{1}{4} - \frac{x}{2} \right ) e^{-2x} + C \)

q4606r4\\
\(\displaystyle -\left ( \frac{x}{2} + \frac{1}{4} \right ) e^{-2x} + C \)

q4606\\
\(\displaystyle \text{Evaluate } \int x e^{-2x}\,dx. \)

q4607r2\\
\(\displaystyle x^2 \ln x + 2x \ln x - 2x + C \)

q4607r3\\
\(\displaystyle 2x \ln x + 2x + C \)

q4607r4\\
\(\displaystyle e^{(x^2 + 2x)} + C \)

q4607\\
\(\displaystyle \text{Evaluate } \int \ln^2 x \, dx. \)

q4608r1\\
\(\displaystyle -2e^{-x} \cos 2x \sin 2x + C \)

q4608r2\\
\(\displaystyle 2e^{-x} \cos 2x + C \)

q4608r3\\
\(\displaystyle \frac{e^{-x}}{5} (2\sin 2x - \cos 2x) + C \)

q4608r4\\
\(\displaystyle \ln^2 (\sin 2x) + \cos 2x + C \)

q4608\\
\(\displaystyle \text{Evaluate } \int e^{-x} \cos 2x \, dx. \)

q4609r1\\
\(\displaystyle 2\ln4 - 1/16 \)

q4609r2\\
\(\displaystyle 8 + \ln 2 \)

q4609r3\\
\(\displaystyle 8 \ln 2 + 13/8 \)

q4609r4\\
\(\displaystyle 4 \ln 2 - 15/16 \)

q4609\\
\(\displaystyle \text{Evaluate the definite integral } \int_1^2 x^3 \ln x \, dx. \)

q4610r1\\
\(\displaystyle \pi^2/8 - 1/2 \)

q4610r2\\
\(\displaystyle \pi^2/4 - 1/2 \)

q4610r3\\
\(\displaystyle \pi^2/4 \)

q4610r4\\
\(\displaystyle 0 \)

q4610\\
\(\displaystyle \text{Evaluate the definite integral } \int_0^{\pi/2} x^2 \sin 2x\,dx. \)

q4611r1\\
\(\displaystyle \frac{-x^3}{9} e^{-x^3}+ C \)

q4611r2\\
\(\displaystyle \frac{-x^3}{3}e^{-x^3} + C \)

q4611r3\\
\(\displaystyle \frac{-1}{9} e^{-x^3} + C \)

q4611r4\\
\(\displaystyle \frac{-1}{3}e^{-x^3} + C \)

q4611\\
\(\displaystyle \text{Evaluate } \int x^2 e^{-x^3} \,dx. \)

q4612r1\\
\(\displaystyle -xe^{-x} -e^{-x} + C \)

q4612r2\\
\(\displaystyle \frac{-x^2}{2} e^{-x} + C \)

q4612r3\\
\(\displaystyle \ln x + C \)

q4612r4\\
\(\displaystyle -e^{-x} + \ln x^2 + C \)

q4612\\
\(\displaystyle \text{Evaluate } \int x e^{-x}\,dx. \)

q4613r1\\
\(\displaystyle \frac{1}{240} x(x+5)^{15} - \frac{1}{15}(x+5)^{16} + C \)

q4613r2\\
\(\displaystyle \frac{1}{15} x(x+5)^{15} - \frac{1}{240}(x+5)^{14} + C \)

q4613r3\\
\(\displaystyle \frac{1}{15}x(x+5)^{15} - \frac{1}{240}(x+5)^{16} + C \)

q4613r4\\
\(\displaystyle \frac{x^2(x+5)^{15}}{17} + C \)

q4613\\
\(\displaystyle \text{Evaluate } \int x(x+5)^{14} \,dx. \)

q4614r1\\
\(\displaystyle 1/8 - 3e^4 / 8 \)

q4614r2\\
\(\displaystyle e \)

q4614r3\\
\(\displaystyle e^2 \ln x \)

q4614r4\\
\(\displaystyle 3e^4/8 + 1/8 \)

q4614\\
\(\displaystyle \text{Evaluate the definite integral } \int_1^{e^2} x\ln (x^{1/2}) \, dx. \)

q4615r1\\
\(\displaystyle 0 \)

q4615r2\\
\(\displaystyle 1/4\pi - 2/\pi^2 \)

q4615r3\\
\(\displaystyle 2\pi - 4\pi^2 \)

q4615r4\\
\(\displaystyle 1/2\pi - 1/\pi^2 \)

q4615\\
\(\displaystyle \text{Evaluate the definite integral } \int_0^{1/2} x \cos \pi x\, dx.\)

q4616r1\\
\(\displaystyle 1/2 \)

q4616r2\\
\(\displaystyle \infty \)

q4616r3\\
\(\displaystyle 0 \)

q4616r4\\
\(\displaystyle 1 \)

q4616\\
\(\displaystyle \lim_{x \rightarrow 0^+} \frac{\sin x}{x^{1/2}} \)

q4617r1\\
\(\displaystyle 1 \)

q4617r2\\
\(\displaystyle -1/2 \)

q4617r3\\
\(\displaystyle 1/2 \)

q4617r4\\
\(\displaystyle 0 \)

q4617\\
\(\displaystyle \lim_{x \rightarrow 0} \frac{\sin x - x}{\tan x - x} \)

q4618r1\\
\(\displaystyle \infty \)

q4618r2\\
\(\displaystyle \ln(5/3) \)

q4618r3\\
\(\displaystyle 2 \)

q4618r4\\
\(\displaystyle 0 \)

q4618\\
\(\displaystyle \lim_{x \rightarrow 0} \frac{5^x - 3^x}{x} \)

q4619r1\\
\(\displaystyle 0 \)

q4619r2\\
\(\displaystyle \infty \)

q4619r3\\
\(\displaystyle 1/2 \)

q4619r4\\
\(\displaystyle 1 \)

q4619\\
\(\displaystyle \lim_{x \rightarrow 1} \frac{\ln x}{x-1} \)

q4620r1\\
\(\displaystyle -1/a^2 \)

q4620r2\\
\(\displaystyle -1/a \)

q4620r3\\
\(\displaystyle 1/a^2 \)

q4620r4\\
\(\displaystyle 0 \)

q4620\\
\(\displaystyle \lim_{x \rightarrow a} \frac{x^{-1} - a^{-1}}{x-a} \)

q4621r1\\
\(\displaystyle 0 \)

q4621r2\\
\(\displaystyle \infty \)

q4621r3\\
\(\displaystyle -\infty \)

q4621r4\\
\(\displaystyle 1/3 \)

q4621\\
\(\displaystyle \lim_{x \rightarrow \infty} \frac{e^x}{x^3} \)

q4622r1\\
\(\displaystyle 0 \)

q4622r2\\
\(\displaystyle 1 \)

q4622r3\\
\(\displaystyle \infty \)

q4622r4\\
\(\displaystyle 1/2 \)

q4622\\
\(\displaystyle \lim_{x \rightarrow 0} \frac{x - \tan x}{1 - \cos x} \)

q4623r1\\
\(\displaystyle 1 \)

q4623r2\\
\(\displaystyle \infty \)

q4623r3\\
\(\displaystyle 1/2 \)

q4623r4\\
\(\displaystyle 0 \)

q4623\\
\(\displaystyle \lim_{x \rightarrow \pi/2^-} [ \sec x - \tan x] \)

q4624r1\\
\(\displaystyle e \)

q4624r2\\
\(\displaystyle \infty \)

q4624r3\\
\(\displaystyle 0 \)

q4624r4\\
\(\displaystyle 1 \)

q4624\\
\(\displaystyle \lim_{x \rightarrow e} \frac{\ln (\ln x)}{\ln x - 1} \)

q4625r1\\
\(\displaystyle \infty \)

q4625r2\\
\(\displaystyle 8/9 \)

q4625r3\\
\(\displaystyle 1 \)

q4625r4\\
\(\displaystyle 1/2 \)

q4625\\
\(\displaystyle \lim_{x \rightarrow 4} \frac{x^2 - 16}{x^2 + x - 20} \)

q4626r1\\
\(\displaystyle 1 \)

q4626r2\\
\(\displaystyle \infty \)

q4626r3\\
\(\displaystyle 4/5 \)

q4626r4\\
\(\displaystyle \ln(4/5) \)

q4626\\
\(\displaystyle \lim_{x \rightarrow 1} \frac{x^2 + 2x - 3}{x^2 + 3x - 4} \)

q4627r1\\
\(\displaystyle -1/2 \)

q4627r2\\
\(\displaystyle -1 \)

q4627r3\\
\(\displaystyle -\infty \)

q4627r4\\
\(\displaystyle -1/3 \)

q4627\\
\(\displaystyle \lim_{x \rightarrow 0} \frac{xe^{3x}}{1-e^{3x}} \)

q4628r1\\
\(\displaystyle 0 \)

q4628r2\\
\(\displaystyle \infty \)

q4628r3\\
\(\displaystyle 1 \)

q4628r4\\
\(\displaystyle -1 \)

q4628\\
\(\displaystyle \lim_{x \rightarrow \infty} \frac{4x^3 - 2x + 1}{4x^3 + 2} \)

q4629r1\\
\(\displaystyle 0 \)

q4629r2\\
\(\displaystyle 1 \)

q4629r3\\
\(\displaystyle \infty \)

q4629r4\\
\(\displaystyle 1/2 \)

q4629\\
\(\displaystyle \lim_{x \rightarrow 0^+} [\sin x \ln x] \)

q4630r1\\
\(\displaystyle \infty \)

q4630r2\\
\(\displaystyle 0 \)

q4630r3\\
\(\displaystyle 1 \)

q4630r4\\
\(\displaystyle 1/2 \)

q4630\\
\(\displaystyle \lim_{x \rightarrow 0} (\sin x)^x \)

q4631r1\\
\(\displaystyle 1 \)

q4631r2\\
\(\displaystyle \infty \)

q4631r3\\
\(\displaystyle -\infty \)

q4631r4\\
\(\displaystyle 0 \)

q4631\\
\(\displaystyle \lim_{x \rightarrow 0^+} x \ln (\sin x) \)

q4673r1\\
\(\displaystyle \Delta x \)

q4673r2\\
\(\displaystyle 1 \)

q4673r3\\
\(\displaystyle 0 \)

q4673r4\\
\(\displaystyle 2\Delta x \)

q4673\\
\(\displaystyle \frac{d}{dx}[0] = 0. \)

q4674r1\\
\(\displaystyle 5 + \Delta x \)

q4674r2\\
\(\displaystyle 5 - \Delta x \)

q4674r3\\
\(\displaystyle 5 \)

q4674r4\\
\(\displaystyle 5 + 2\Delta x \)

q4674\\
\(\displaystyle \frac{d}{dx}[6 + 5x] = 5. \)

q4675r1\\
\(\displaystyle 4x \)

q4675r2\\
\(\displaystyle 4x + 2\Delta x \)

q4675r3\\
\(\displaystyle 4x + \Delta x \)

q4675r4\\
\(\displaystyle 4x + 4\Delta x \)

q4675\\
\(\displaystyle \frac{d}{dx} [4-2x^2] = -4x. \)

q4676r1\\
\(\displaystyle \frac{-4\Delta x}{4x - 5} \)

q4676r2\\
\(\displaystyle \frac{-5}{(4x-5)(4\Delta x + 4x - 5)} \)

q4676r3\\
\(\displaystyle \frac{-12}{4\Delta x + 4x - 5} \)

q4676r4\\
\(\displaystyle \frac{-12}{(4x-5)(4\Delta x + 4x - 5)} \)

q4676\\
\(\displaystyle \frac{d}{dx} \left [ \frac{3}{4x-5} \right ] = \frac{-12}{(4x-5)^2} \)

q4677r1\\
\(\displaystyle \frac{1}{(4x+4\Delta x)^{1/2} + (4x)^{1/2}} \)

q4677r2\\
\(\displaystyle \frac{\Delta x}{(4x + 4 \Delta x)^{1/2} + (4x)^{1/2}} \)

q4677r3\\
\(\displaystyle \frac{4}{(4x+4\Delta x)^{1/2} + (4x)^{1/2}} \)

q4677r4\\
\(\displaystyle \frac{4}{(4x + 4\Delta x)^{1/2}} \)

q4677\\
\(\displaystyle \frac{d}{dx} [(4x)^{1/2}] = \frac{1}{2(4x)^{1/2}} \)

q4678r1\\
\(\displaystyle 0 \)

q4678r2\\
\(\displaystyle \Delta x \)

q4678r3\\
\(\displaystyle 2\Delta x \)

q4678r4\\
\(\displaystyle 1 \)

q4678\\
\(\displaystyle \frac{d}{dx}[-2] = 0. \)

q4679r1\\
\(\displaystyle -10 + \Delta x \)

q4679r2\\
\(\displaystyle -10 \)

q4679r3\\
\(\displaystyle -10-\Delta x \)

q4679r4\\
\(\displaystyle -10+2\Delta x \)

q4679\\
\(\displaystyle \frac{d}{dx} [10-10x] = -10. \)

q4680r1\\
\(\displaystyle -6x + \Delta x \)

q4680r2\\
\(\displaystyle -6x \)

q4680r3\\
\(\displaystyle -6x - 3\Delta x \)

q4680r4\\
\(\displaystyle -6x + 6\Delta x \)

q4680\\
\(\displaystyle \frac{d}{dx}[6 - 3x^2] = -6x. \)

q4681r1\\
\(\displaystyle \frac{-8}{(4\Delta x + 4x + 6)(4x + 6)} \)

q4681r2\\
\(\displaystyle \frac{-4}{(4\Delta x + 4x + 6)(4x + 6)} \)

q4681r3\\
\(\displaystyle \frac{-8}{4\Delta x + 4x + 6} \)

q4681r4\\
\(\displaystyle \frac{8}{(4\Delta x + 4x + 6)(4x + 6)} \)

q4681\\
\(\displaystyle \frac{d}{dx} \left [ \frac{2}{4x + 6} \right ] = \frac{-8}{(4x + 6)^2} \)

q4682r1\\
\(\displaystyle 4x + 3 \)

q4682r2\\
\(\displaystyle 4x + \Delta x + 3 \)

q4682r3\\
\(\displaystyle 4x + 2\Delta x + 3 \)

q4682r4\\
\(\displaystyle 4x + 4\Delta x + 3 \)

q4682\\
\(\displaystyle \frac{d}{dx} [2x^2 + 3x] = 4x+3. \)

q4695r1\\
\(\displaystyle \frac{2}{x^8} \)

q4695r2\\
\(\displaystyle \frac{2}{x^4} \)

q4695r3\\
\(\displaystyle \frac{-2}{x^5} \)

q4695r4\\
\(\displaystyle \frac{-8}{x^5} \)

q4695\\
\(\displaystyle \text{Differentiate } f(x) = \frac{2}{x^4}. \)

q4697r1\\
\(\displaystyle \frac{20}{x^3} - \frac{4}{x^5} \)

q4697r2\\
\(\displaystyle \frac{1}{x^2} \)

q4697r3\\
\(\displaystyle \frac{-4}{x^3} + \frac{20}{x^5} \)

q4697r4\\
\(\displaystyle \frac{4x-1}{x^8} \)

q4697\\
\(\displaystyle \text{Differentiate } f(x) = \frac{2x^2 - 5}{x^4} \)

q4698r1\\
\(\displaystyle \frac{1}{\tan^2 x} \)

q4698r2\\
\(\displaystyle \frac{-\sec x}{\tan x} \)

q4698r3\\
\(\displaystyle \frac{\sec x (1- \sec x)}{\tan^2 x} \)

q4698r4\\
\(\displaystyle \frac{\sec x (\sec x - 1)}{\tan^2 x} \)

q4698\\
\(\displaystyle \text{Differentiate } f(x) = \frac{1-\sec x}{\tan x} \)

q4699r1\\
\(\displaystyle 1 \)

q4699r2\\
\(\displaystyle 0 \)

q4699r3\\
\(\displaystyle 1/2 \)

q4699r4\\
\(\displaystyle -1 \)

q4699\\
\(\displaystyle \text{If } f(x) = \frac{1-x^2}{1+x^2}, \text{ what is } f'(1)? \)

q4700r1\\
\(\displaystyle 10/3 \)

q4700r2\\
\(\displaystyle 8/3 \)

q4700r3\\
\(\displaystyle 7/2 \)

q4700r4\\
\(\displaystyle 20/9 \)

q4700\\
\(\displaystyle \text{If } h(0) = 3 \text{ and } h'(0) = 2, \text{ find } f'(0) \text{ if } f(x) = h(x) - \frac{1}{h(x)}. \)

q4701r1\\
\(\displaystyle \frac{3x^4 + 5}{(x-1)^3} \)

q4701r2\\
\(\displaystyle \frac{(9x^4 - 12x^3 - 5}{(x-1)^2} \)

q4701r3\\
\(\displaystyle \frac{3x^4 + 5}{(x-1)^2} \)

q4701r4\\
\(\displaystyle \frac{12x^3}{(x-1)^2} \)

q4701\\
\(\displaystyle \text{Differentiate } f(x) = \frac{3x^4 + 5}{x-1}. \)

q4702r1\\
\(\displaystyle \frac{3x^2 - 2x^3}{(1-x)^2} \)

q4702r2\\
\(\displaystyle \frac{2x^3 - 3x^2}{(1-x)^4} \)

q4702r3\\
\(\displaystyle \frac{2x^3 - 3x^2}{(1-x)^2} \)

q4702r4\\
\(\displaystyle \frac{3x^2}{(1-x)^2} \)

q4702\\
\(\displaystyle \text{Differentiate } f(x) = \frac{x^3}{1-x} \)

q4703r1\\
\(\displaystyle \frac{a}{c} \)

q4703r2\\
\(\displaystyle \frac{ad + bc}{(c+d)^2} \)

q4703r3\\
\(\displaystyle \frac{ad - bc}{(c+d)^2} \)

q4703r4\\
\(\displaystyle \frac{a}{(c+d)^2} \)

q4703\\
\(\displaystyle \text{Find } f'(1) \text{ if } f(x) = \frac{ax+b}{cx+d}. \)

q4704r1\\
\(\displaystyle \frac{-6}{x^3} \)

q4704r2\\
\(\displaystyle \frac{x^2 + 3}{x^2} \)

q4704r3\\
\(\displaystyle \frac{-6}{x^2} \)

q4704r4\\
\(\displaystyle 0 \)

q4704\\
\(\displaystyle \text{What is the second derivative of } f(x) = \frac{x^2 - 3}{x}? \)

q4705r1\\
\(\displaystyle \frac{e^x}{(e^x + 1)^4} \)

q4705r2\\
\(\displaystyle \frac{2e^x}{(e^x - 1)^2} \)

q4705r3\\
\(\displaystyle \frac{2e^x}{(e^x + 1)^2} \)

q4705r4\\
\(\displaystyle \frac{e^x}{(e^x + 1)^2} \)

q4705\\
\(\displaystyle \text{Differentiate } f(x) = \frac{e^x - 1}{e^x + 1}. \)

q4706r1\\
\(\displaystyle 9x^2 + 6x - 2 \)

q4706r2\\
\(\displaystyle 6x^3 + 2x^2 - 3x \)

q4706r3\\
\(\displaystyle 18x^2 - 3x + 4 \)

q4706r4\\
\(\displaystyle 18x^2 + 4x - 3 \)

q4706\\
\(\displaystyle \text{Differentiate } f(x) = (2x^2-1)(3x+1) \)

q4707r1\\
\(\displaystyle 1 + \frac{4}{x^2} \)

q4707r2\\
\(\displaystyle \frac{-2}{x^2} - \frac{4}{x^3} - \frac{12}{x^4} \)

q4707r3\\
\(\displaystyle -2x^2 - 4x^3 - 12x^4 \)

q4707r4\\
\(\displaystyle \frac{-x^2}{2} - \frac{x^3}{4} - \frac{x^4}{12} \)

q4707\\
\(\displaystyle \text{Differentiate } f(x) = \left ( 1 + \frac{2}{x} \right ) \left ( 1 + \frac{2}{x^2} \right ). \)

q4708r1\\
\(\displaystyle 2x \sin \frac{1}{x} + \cos \frac{1}{x} \)

q4708r2\\
\(\displaystyle 2x\sin \frac{1}{x} \)

q4708r3\\
\(\displaystyle 2x \sin \frac{1}{x} - \cos \frac{1}{x} \)

q4708r4\\
\(\displaystyle 2x \cos \frac{1}{x} + \sin \frac{1}{x} \)

q4708\\
\(\displaystyle \text{Differentiate } f(x) = x^2 \cos \left ( \frac{1}{x} \right ). \)

q4710r1\\
\(\displaystyle 2 \)

q4710r2\\
\(\displaystyle -1 \)

q4710r3\\
\(\displaystyle 1 \)

q4710r4\\
\(\displaystyle 0 \)

q4710\\
\(\displaystyle \text{If } f(x) = x^2(x+1), \text{ what is } f'(0)? \)

q4711r1\\
\(\displaystyle 2 \)

q4711r2\\
\(\displaystyle 3 \)

q4711r3\\
\(\displaystyle -2 \)

q4711r4\\
\(\displaystyle 0 \)

q4711\\
\(\displaystyle \text{If } h(0) = 3 \text{ and } h'(0) = 2, \text{ find } f'(0) \text{ for } f(x) = xh(x). \)

q4712r1\\
\(\displaystyle x^3 - 3x^2 - x \)

q4712r2\\
\(\displaystyle 2x \)

q4712r3\\
\(\displaystyle 3x^2 - 6x - 1 \)

q4712r4\\
\(\displaystyle 6x - 6 \)

q4712\\
\(\displaystyle \text{Differentiate } f(x) = (x^2-1)(x-3). \)

q4713r1\\
\(\displaystyle -40x^9 + \frac{81}{2} x^8 - 32x^7 + \frac{63}{2} x^6 \)

q4713r2\\
\(\displaystyle -80x^9 + 81x^8 - \ln (17x^2) \)

q4713r3\\
\(\displaystyle -80x^9 + 81x^8 - 64x^7 + 63x^6 \)

q4713r4\\
\(\displaystyle 72x^7 - 72x^8 + \ln(17x^2) \)

q4713\\
\(\displaystyle \text{Differentiate } f(x) = (9x^8 - 8x^9) \left ( x + \frac{1}{x} \right ). \)

q4714r1\\
\(\displaystyle 140 \)

q4714r2\\
\(\displaystyle 35 \)

q4714r3\\
\(\displaystyle 70 \)

q4714r4\\
\(\displaystyle 20 \)

q4714\\
\(\displaystyle \text{If } f(x) = 10x^4(3x^2 - 1), \text{ what is } f'(1)? \)

q4715r1\\
\(\displaystyle 37 \)

q4715r2\\
\(\displaystyle -4 \)

q4715r3\\
\(\displaystyle 4 \)

q4715r4\\
\(\displaystyle -3 \)

q4715\\
\(\displaystyle \text{If } h(0) = 4 \text{ and } h'(0) = 3, \text{ find } f'(0) \text{ for } f(x) = 2x^2 h(x) - 3x \)

q4716r1\\
\(\displaystyle 3\sec^2(e^{3x}) \)

q4716r2\\
\(\displaystyle 3e^{3x} \sec^2 (e^{3x}) \)

q4716r3\\
\(\displaystyle \sec^2 (e^{3x}) \)

q4716r4\\
\(\displaystyle 9e^{3x} \sec^2(e^{3x}) \)

q4716\\
\(\displaystyle \text{Differentiate } f(x) = \tan(e^{3x}) \)

q4717r1\\
\(\displaystyle \pi/10 \)

q4717r2\\
\(\displaystyle \pi/12 \)

q4717r3\\
\(\displaystyle \pi/15 \)

q4717r4\\
\(\displaystyle \pi/18 \)

q4717\\
\(\displaystyle \text{Express } 10^{\circ} \text{ in radians.} \)

q4718r1\\
\(\displaystyle 24^{\circ} \)

q4718r2\\
\(\displaystyle 30^{\circ} \)

q4718r3\\
\(\displaystyle 36^{\circ} \)

q4718r4\\
\(\displaystyle 40^{\circ} \)

q4718\\
\(\displaystyle \text{Express } \pi/5 \text{ in degrees.} \)

q4719r1\\
\(\displaystyle 4 \)

q4719r2\\
\(\displaystyle 3 \)

q4719r3\\
\(\displaystyle 2 \)

q4719r4\\
\(\displaystyle 1 \)

q4719\\
\(\displaystyle \text{Evaluate } \log_3 6 + \frac{3}{2} \log_3 36 - 4\log_3 2 \)

q4720r1\\
\(\displaystyle x^4 \)

q4720r2\\
\(\displaystyle 2x^2 \)

q4720r3\\
\(\displaystyle 2x^4 \)

q4720r4\\
\(\displaystyle 8 \)

q4720\\
\(\displaystyle \text{Evaluate } 2e^{4\ln x}. \)

q4721r1\\
\(\displaystyle 0 \)

q4721r2\\
\(\displaystyle 2 \)

q4721r3\\
\(\displaystyle e^2 \)

q4721r4\\
\(\displaystyle \infty \)

q4721\\
\(\displaystyle \text{Evaluate } \lim_{x \rightarrow \infty} \frac{2+e^{-x}}{e^{-2x}} \)

q4722r1\\
\(\displaystyle \pi/3, 2\pi/3 \)

q4722r2\\
\(\displaystyle \pi/6, 5\pi/6 \)

q4722r3\\
\(\displaystyle \pi/6, 11\pi/6 \)

q4722r4\\
\(\displaystyle \pi/3, 5\pi/3 \)

q4722\\
\(\displaystyle \text{Find all solutions of } \sec q = 2 \text{ for } 0 \leq q < 2\pi \)

q4723r1\\
\(\displaystyle 2 \)

q4723r2\\
\(\displaystyle 4 \)

q4723r3\\
\(\displaystyle 5 \)

q4723r4\\
\(\displaystyle 10 \)

q4723\\
\(\displaystyle \text{Solve } \log(x^3 - x^2) = 2. \)

q4724r1\\
\(\displaystyle 0.287 \)

q4724r2\\
\(\displaystyle 0.536 \)

q4724r3\\
\(\displaystyle 1.33 \)

q4724r4\\
\(\displaystyle \text{No real solutions} \)

q4724\\
\(\displaystyle \text{Solve } 3e^{-x^2} = 4. \text{ Approximate your answer to the nearest hundreth.} \)

q4725r1\\
\(\displaystyle 11\pi/12 \)

q4725r2\\
\(\displaystyle 14\pi/15 \)

q4725r3\\
\(\displaystyle 17\pi/18 \)

q4725r4\\
\(\displaystyle 19\pi/20 \)

q4725\\
\(\displaystyle \text{Express } 170^{\circ} \text{ in radians.} \)

q4726r1\\
\(\displaystyle 46^{\circ} \)

q4726r2\\
\(\displaystyle 54^{\circ} \)

q4726r3\\
\(\displaystyle 60^{\circ} \)

q4726r4\\
\(\displaystyle 64^{\circ} \)

q4726\\
\(\displaystyle \text{Express } 3\pi/10 \text{ radians in degrees.} \)

q4727r1\\
\(\displaystyle 0 \)

q4727r2\\
\(\displaystyle -3 \)

q4727r3\\
\(\displaystyle 3 \)

q4727r4\\
\(\displaystyle 2 \)

q4727\\
\(\displaystyle \text{Evaluate } 2\log_2 3 - 2\log_2 6 + \log_2 4 \)

q4728r1\\
\(\displaystyle \frac{x^6}{3} \)

q4728r2\\
\(\displaystyle \frac{3x^2}{2} \)

q4728r3\\
\(\displaystyle 2 + \frac{x^2}{3} \)

q4728r4\\
\(\displaystyle \frac{3}{x^2} \)

q4728\\
\(\displaystyle \text{Evaluate } 3\ln e^{x^2/2} \)

q4729r1\\
\(\displaystyle 0 \)

q4729r2\\
\(\displaystyle 2 \)

q4729r3\\
\(\displaystyle -\infty \)

q4729r4\\
\(\displaystyle \infty \)

q4729\\
\(\displaystyle \text{Evaluate } \lim_{x \rightarrow 1} \frac{-2}{\ln x} \)

q4730r1\\
\(\displaystyle \pi/4, 3\pi/4 \)

q4730r2\\
\(\displaystyle 3\pi/4, 5\pi/4 \)

q4730r3\\
\(\displaystyle \pi/4, 7\pi/4 \)

q4730r4\\
\(\displaystyle 5\pi/4, 7\pi/4 \)

q4730\\
\(\displaystyle \text{Find all solutions of } \csc q = -2^{1/2} \text{ for } 0 \leq q < 2\pi. \)

q4731r1\\
\(\displaystyle 0 \)

q4731r2\\
\(\displaystyle -2 \)

q4731r3\\
\(\displaystyle 2 \)

q4731r4\\
\(\displaystyle \text{No real solutions} \)

q4731\\
\(\displaystyle \text{Solve } \log(x^2 - 2) - \log x = 0. \)

q4732r1\\
\(\displaystyle -0.511 \)

q4732r2\\
\(\displaystyle -0.128 \)

q4732r3\\
\(\displaystyle 0.128 \)

q4732r4\\
\(\displaystyle 0.5111 \)

q4732\\
\(\displaystyle \text{Solve } 5e^{4x} = 3. \text{ Approximate your answer to the nearest hundreth.} \)

q4733\\
\(\displaystyle (x+y)^{1/2} = x^{1/2} + y^{1/2} \)

q4734\\
\(\displaystyle \frac{d}{dx} \cos x = -\sin x \)

q4736\\
\(\displaystyle \frac{d}{dx} \ln 2 = 0 \)

q4737\\
\(\displaystyle \int x^3\,dx = \frac{x^4}{4} \)

q4738\\
\(\displaystyle \frac{\sin(4x)}{\cos(2x)} = \tan(2x) \)

q4739\\
\(\displaystyle (x+5)^2 = x^2 + 25 \)

q4740\\
\(\displaystyle \text{If } x^2 + 3 < y^2 + 3, \text{ then } x^2 <  y^2. \)

q4741\\
\(\displaystyle \int \frac{1}{x}\,dx = \ln |x| + C \)

q4742\\
\(\displaystyle \int \frac{1}{x^2}\,dx = \frac{-3}{x^3} + C \)

q4743\\
\(\displaystyle \frac{d}{dx} \sin(x^3) = 3x^2 \cos(x^3) \)

q4744\\
\(\displaystyle \frac{d}{dx} \sin(3\pi) = 3\cos(3 \pi) \)

q4745\\
\(\displaystyle \lim_{x \rightarrow 0} \frac{x^3-x}{x^2+5} = 0 \)

q4746\\
\(\displaystyle \frac{1}{x+4} = \frac{1}{x} + \frac{1}{4} \)

q4747\\
\(\displaystyle \text{If } x^2 + x = x^3 - x, \text{ then } x = 2 \text{ or } x = -1. \)

q4748\\
\(\displaystyle 36x^2 - 9 = (6x-3)(6x+3) \)

q4749\\
\(\displaystyle \int \frac{(x-2)^6}{(-2+x)^2}\,dx = \frac{(x-2)^5}{5} + C \)

q4750\\
\(\displaystyle x + \frac{1}{x} = \frac{x^2 + 1}{x} \)

q4751\\
\(\displaystyle (4x^2 + 9x^4)^{1/2} = 2x + 3x^2 \)

q4752\\
\(\displaystyle \frac{d}{dx} 5^x = x5^{x-1} \)

q4753\\
\(\displaystyle (13x^2 + 12x^2)^{1/2} = 5x \)

q4764r1\\
\(\displaystyle (-\infty, -1) \)

q4764r2\\
\(\displaystyle (-1, 2) \)

q4764r3\\
\(\displaystyle (2, \infty) \)

q4764r4\\
\(\displaystyle (-1, 5) \)

q4764\\
\(\displaystyle \text{On which intervals is } f(x) \text{ decreasing?} \)

q4765r1\\
\(\displaystyle x < -2 \)

q4765r2\\
\(\displaystyle x > -2 \)

q4765r3\\
\(\displaystyle x > 2 \)

q4765r4\\
\(\displaystyle x < 2 \)

q4765\\
\(\displaystyle \text{Where is } f(x) \text{ concave up?} \)

q4766r1\\
\(\displaystyle (-2, -80) \)

q4766r2\\
\(\displaystyle (1, -23) \)

q4766r3\\
\(\displaystyle (2, -80) \)

q4766r4\\
\(\displaystyle (0,0) \)

q4766\\
\(\displaystyle \text{Find all inflection points of } f(x). \)

q4768r1\\
\(\displaystyle x = 12^{1/2} \)

q4768r2\\
\(\displaystyle x = 0 \)

q4768r3\\
\(\displaystyle x = 12^{1/2} \)

q4768r4\\
\(\displaystyle x = -2 \)

q4768\\
\(\displaystyle \text{At what value(s) of } x \text{ does } f(x) \text{ have a relative maximum?} \)

q4769r1\\
\(\displaystyle x = -12^{1/2} \)

q4769r2\\
\(\displaystyle x = 0 \)

q4769r3\\
\(\displaystyle x = 12^{1/2} \)

q4769r4\\
\(\displaystyle x = -2 \)

q4769\\
\(\displaystyle \text{At what value(s) of } x \text{ does } f(x) \text{ have a relative minimum?} \)

q4770r1\\
\(\displaystyle \text{A} \)

q4770r2\\
\(\displaystyle \text{B} \)

q4770r3\\
\(\displaystyle \text{C} \)

q4770r4\\
\(\displaystyle \text{D} \)

q4770\\
\(\displaystyle \text{Qualitatively, which plot below could be the graph of } y = f(x)? \)

q4771r1\\
\(\displaystyle x < 5/4 \)

q4771r2\\
\(\displaystyle x > 5/4 \)

q4771r3\\
\(\displaystyle x < 4 \)

q4771r4\\
\(\displaystyle x > 4 \)

q4771\\
\(\displaystyle \text{Where is } f(x) \text{ concave down}? \)

q4772r1\\
\(\displaystyle (0, 10) \)

q4772r2\\
\(\displaystyle (4, -46) \)

q4772r3\\
\(\displaystyle (5/4, -225/8) \)

q4772r4\\
\(\displaystyle (-5/4, -225/8) \)

q4772\\
\(\displaystyle \text{Find all inflection points of } f(x). \)

q4773r1\\
\(\displaystyle x = -1/2 \)

q4773r2\\
\(\displaystyle x = -3 \)

q4773r3\\
\(\displaystyle x = 1/2 \)

q4773r4\\
\(\displaystyle x = 3 \)

q4773\\
\(\displaystyle \text{At what value(s) of } x \text{ does } f(x) \text{ have a relative maximum?} \)

q4774r1\\
\(\displaystyle x = -1/2 \)

q4774r2\\
\(\displaystyle x = -3 \)

q4774r3\\
\(\displaystyle x = 1/2 \)

q4774r4\\
\(\displaystyle x = 3 \)

q4774\\
\(\displaystyle \text{At what values of } x \text{ does } f(x) \text{ have a relative minimum?} \)

q4775r1\\
\(\displaystyle \text{A} \)

q4775r2\\
\(\displaystyle \text{B} \)

q4775r3\\
\(\displaystyle \text{C} \)

q4775r4\\
\(\displaystyle \text{D} \)

q4775\\
\(\displaystyle \text{Qualitatively, which plot below could be the graph of } y = f(x)? \)

q4776r1\\
\(\displaystyle -1 \)

q4776r2\\
\(\displaystyle -2 \)

q4776r3\\
\(\displaystyle 0 \)

q4776r4\\
\(\displaystyle 2 \)

q4776\\
\(\displaystyle \text{Find the critical points of } f(x). \)

q4777r1\\
\(\displaystyle (0, 5) \)

q4777r2\\
\(\displaystyle (2, -15) \)

q4777r3\\
\(\displaystyle (-1, 12) \)

q4777r4\\
\(\displaystyle \text{There are none.} \)

q4777\\
\(\displaystyle \text{Find the location of relative maxima of } f(x). \)

q4778r1\\
\(\displaystyle (0, 5) \)

q4778r2\\
\(\displaystyle (2, -15) \)

q4778r3\\
\(\displaystyle (-1, 12) \)

q4778r4\\
\(\displaystyle \text{There are none.} \)

q4778\\
\(\displaystyle \text{Find the location of relative minima of } f(x). \)

q4779r1\\
\(\displaystyle -15 \)

q4779r2\\
\(\displaystyle 14 \)

q4779r3\\
\(\displaystyle 37 \)

q4779r4\\
\(\displaystyle 5 \)

q4779\\
\(\displaystyle \text{What is the maximum value of } f(x) \text{ on the interval } [0, 4]? \)

q4780r1\\
\(\displaystyle 14 \)

q4780r2\\
\(\displaystyle 37 \)

q4780r3\\
\(\displaystyle -15 \)

q4780r4\\
\(\displaystyle -23 \)

q4780\\
\(\displaystyle \text{What is the minimum value of } f(x) \text{ on the interval } [0, 4]? \)

q4781r1\\
\(\displaystyle A \)

q4781r2\\
\(\displaystyle B \)

q4781r3\\
\(\displaystyle C \)

q4781r4\\
\(\displaystyle D \)

q4781\\
\(\displaystyle \text{Which of the above could be the graph of } f(x)? \)

q4783\\
\(\displaystyle \lim_{x \rightarrow 3} \frac{x^2-9}{x-3} = 6 \)

q4784\\
\(\displaystyle \frac{d}{dx} 4^x = 4^x \ln 4 \)

q4785\\
\(\displaystyle \int (x^3 - 2)\,dx = \frac{1}{4}x^4 - 2x \)

q4786\\
\(\displaystyle \frac{1}{3} + \frac{1}{y} = \frac{1}{3+y} \)

q4787\\
\(\displaystyle \frac{d}{dx} \left [ \frac{g(x)}{h(x)} \right ] = \frac{h(x)g'(x) - g(x)h'(x)}{h^2(x)} \)

q4788\\
\(\displaystyle \frac{\cos(5x)}{x} = \cos 5 \)

q4789\\
\(\displaystyle \int \frac{3}{x}\,dx = \ln |3x| + C \)

q4790\\
\(\displaystyle (y-3)^2 = y^2 - 6y + 9 \)

q4791\\
\(\displaystyle \frac{d}{dx} \tan(x^3 - 2x) = \sec^2(3x^2-2) \)

q4792\\
\(\displaystyle \text{If } (k+1)x = 3x, \text{ then } k = 2. \)

q4793\\
\(\displaystyle \frac{d}{dx} \sin x = -\cos x \)

q4794\\
\(\displaystyle (x+5)^{1/2} = x^{1/2} + 5^{1/2} \)

q4795\\
\(\displaystyle \frac{d}{dx} \ln 5 = 0 \)

q4796\\
\(\displaystyle \int e^{2x}\,dx = 2e^{2x} + C \)

q4797\\
\(\displaystyle \text{If } x < 4, \text{ then } kx < 4k \text{ for a constant } k. \)

q4798\\
\(\displaystyle (x+2)^3 = x^3 + 2^3 \)

q4799\\
\(\displaystyle \text{If } x^2 - 1 < y^2 - 1, \text{ then } x < y. \)

q4800\\
\(\displaystyle \frac{1}{x} + \frac{1}{y} = \frac{xy}{x+y} \)

q4801\\
\(\displaystyle \frac{d}{dx} [ \cos(5x - 6) ] = \sin(5x - 6) + \cos 5 \)

q4802\\
\(\displaystyle \int \tan x\,dx = \sec^2 x + C \)

q4805r1\\
\(\displaystyle (-\infty, -1/2) \)

q4805r2\\
\(\displaystyle (-1/2, 3/2) \)

q4805r3\\
\(\displaystyle (3/2, \infty) \)

q4805r4\\
\(\displaystyle (-1/2, \infty) \)

q4805\\
\(\displaystyle \text{On which intervals is } f(x) \text{ increasing?} \)

q4806r1\\
\(\displaystyle -1/2 \)

q4806r2\\
\(\displaystyle 2 \)

q4806r3\\
\(\displaystyle -3/2 \)

q4806r4\\
\(\displaystyle 3/2 \)

q4806\\
\(\displaystyle \text{Find the critical points of } f(x). \)

q4807r1\\
\(\displaystyle (-1/2, 5/2) \)

q4807r2\\
\(\displaystyle (3/2, -27/2) \)

q4807r3\\
\(\displaystyle (2, -10) \)

q4807r4\\
\(\displaystyle \text{There are none.} \)

q4807\\
\(\displaystyle \text{Find the location of relative maxima of } f(x). \)

q4808r1\\
\(\displaystyle (-1/2, 5/2) \)

q4808r2\\
\(\displaystyle (3/2, -27/2) \)

q4808r3\\
\(\displaystyle (2, -10) \)

q4808r4\\
\(\displaystyle \text{There are none.} \)

q4808\\
\(\displaystyle \text{Find the location of relative minima of } f(x). \)

q4809r1\\
\(\displaystyle 5/2 \)

q4809r2\\
\(\displaystyle -10 \)

q4809r3\\
\(\displaystyle -27/2 \)

q4809r4\\
\(\displaystyle 10 \)

q4809\\
\(\displaystyle \text{What is the maximum value of } f(x) \text{ on the interval } [-1, 2]? \)

q4810r1\\
\(\displaystyle 5/2 \)

q4810r2\\
\(\displaystyle -10 \)

q4810r3\\
\(\displaystyle -27/2 \)

q4810r4\\
\(\displaystyle -15 \)

q4810\\
\(\displaystyle \text{What is the minimum value of } f(x) \text{ on the interval } [-1, 2]? \)

q4811r1\\
\(\displaystyle A \)

q4811r2\\
\(\displaystyle B \)

q4811r3\\
\(\displaystyle C \)

q4811r4\\
\(\displaystyle D \)

q4811\\
\(\displaystyle \text{Which of the above could be the graph of } f(x)? \)

q4816r1\\
\(\displaystyle 0 \)

q4816r2\\
\(\displaystyle 1 \)

q4816r3\\
\(\displaystyle -1 \)

q4816r4\\
\(\displaystyle \text{Does not exist} \)

q4816\\
\(\displaystyle \text{Evaluate } \lim_{x \rightarrow 0} [ x + \frac{1}{x}]. \)

q4817r1\\
\(\displaystyle 0 \)

q4817r2\\
\(\displaystyle 1/2 \)

q4817r3\\
\(\displaystyle 2 \)

q4817r4\\
\(\displaystyle \text{Does not exist} \)

q4817\\
\(\displaystyle \text{Evaluate } \lim_{x \rightarrow 1} \frac{x^3 - 1}{x+1} \)

q4818r1\\
\(\displaystyle 0 \)

q4818r2\\
\(\displaystyle -1 \)

q4818r3\\
\(\displaystyle 1 \)

q4818r4\\
\(\displaystyle \text{Does not exist} \)

q4818\\
\(\displaystyle \text{Evaluate } \lim_{x \rightarrow -2} \frac{|x|}{x} \)

q4819r1\\
\(\displaystyle -1 \)

q4819r2\\
\(\displaystyle 1 \)

q4819r3\\
\(\displaystyle 0 \)

q4819r4\\
\(\displaystyle \text{Does not exist} \)

q4819\\
\(\displaystyle \text{Let } f(x) = 1+x \text{ for } x < 1, f(x) = 6 \text{ for } x = 1, f(x) = 1 - x \text{ for } x > 1. \text{ Find } \lim_{x \rightarrow 1^+} f(x). \)

q4820r1\\
\(\displaystyle -2 \)

q4820r2\\
\(\displaystyle 2 \)

q4820r3\\
\(\displaystyle 0 \)

q4820r4\\
\(\displaystyle \text{Does not exist} \)

q4820\\
\(\displaystyle \text{Let } f(x) = 1+x \text{ for } x < 1, f(x) = 6 \text{ for } x = 1, f(x) = 1 - x \text{ for } x > 1. \text{ Find } \lim_{x \rightarrow 1^-} f(x). \)

q4821r1\\
\(\displaystyle 0 \)

q4821r2\\
\(\displaystyle 2 \)

q4821r3\\
\(\displaystyle 6 \)

q4821r4\\
\(\displaystyle \text{Does not exist} \)

q4821\\
\(\displaystyle \text{Let } f(x) = 1+x \text{ for } x < 1, f(x) = 6 \text{ for } x = 1, f(x) = 1 - x \text{ for } x > 1. \text{ Find } \lim_{x \rightarrow 1} f(x). \)

q4826r1\\
\(\displaystyle 0 \)

q4826r2\\
\(\displaystyle \frac{5^{1/2}}{5} \)

q4826r3\\
\(\displaystyle -5^{1/2} \)

q4826r4\\
\(\displaystyle \text{Does not exist} \)

q4826\\
\(\displaystyle \text{Evaluate } \lim_{x \rightarrow 1} \frac{(x^2 + 4)^{1/2} - 5^{1/2}}{x-1} \)

q4827r1\\
\(\displaystyle 0 \)

q4827r2\\
\(\displaystyle 2^{1/2} \)

q4827r3\\
\(\displaystyle -2^{1/2} \)

q4827r4\\
\(\displaystyle \text{Does not exist} \)

q4827\\
\(\displaystyle \text{Evaluate } \lim_{x \rightarrow 1^+} \frac{(x-1)^{1/2}}{x} \)

q4828r1\\
\(\displaystyle 0 \)

q4828r2\\
\(\displaystyle 2 \)

q4828r3\\
\(\displaystyle -3 \)

q4828r4\\
\(\displaystyle \text{Does not exist} \)

q4828\\
\(\displaystyle \text{Evaluate } \lim_{x \rightarrow 0} \frac{2x - 5x^2}{x} \)

q4829r1\\
\(\displaystyle 3 \)

q4829r2\\
\(\displaystyle 0 \)

q4829r3\\
\(\displaystyle 1 \)

q4829r4\\
\(\displaystyle \text{Does not exist} \)

q4829\\
\(\displaystyle \text{Let } f(x) = 3x^2 \text{ for } x < 1, f(x) = 5 \text{ for } x = 1, f(x) = 2x^2 - 1 \text{ for } x > 1. \text{ Find } \lim_{x \rightarrow 1^+} f(x). \)

q4830r1\\
\(\displaystyle 3 \)

q4830r2\\
\(\displaystyle 0 \)

q4830r3\\
\(\displaystyle 1 \)

q4830r4\\
\(\displaystyle \text{Does not exist} \)

q4830\\
\(\displaystyle \text{Let } f(x) = 3x^2 \text{ for } x < 1, f(x) = 5 \text{ for } x = 1, f(x) = 2x^2 - 1 \text{ for } x > 1. \text{ Find } \lim_{x \rightarrow 1^-} f(x). \)

q4831r1\\
\(\displaystyle 3 \)

q4831r2\\
\(\displaystyle 0 \)

q4831r3\\
\(\displaystyle 1 \)

q4831r4\\
\(\displaystyle \text{Does not exist} \)

q4831\\
\(\displaystyle \text{Let } f(x) = 3x^2 \text{ for } x < 1, f(x) = 5 \text{ for } x = 1, f(x) = 2x^2 - 1 \text{ for } x > 1. \text{ Find } \lim_{x \rightarrow 1} f(x). \)

q4842r1\\
\(\displaystyle (-\infty, 0), [5, \infty) \)

q4842r2\\
\(\displaystyle (-\infty, \infty), (-\infty, \infty) \)

q4842r3\\
\(\displaystyle (-\infty, 2) \cup (2, 5), (5, 8) \)

q4842r4\\
\(\displaystyle (-\infty, 2), (5, \infty) \)

q4842\\
\(\displaystyle \text{What are the domain and range, respectively, of } f(x) = \frac{1}{(2-x)^{1/2}} + 5? \)

q4843r1\\
\(\displaystyle 6x^2 + 2 \)

q4843r2\\
\(\displaystyle 36x^2 + 24x + 4 \)

q4843r3\\
\(\displaystyle 36x^2 + 4 \)

q4843r4\\
\(\displaystyle 6x^2 + 4 \)

q4843\\
\(\displaystyle \text{If } f(x) = 6x + 2, \text{ what is } f(x^2)? \)

q4844r1\\
\(\displaystyle 14x^2 \sin x \)

q4844r2\\
\(\displaystyle 4x^4 \cos x \)

q4844r3\\
\(\displaystyle x^4 - 7x^2 \)

q4844r4\\
\(\displaystyle x^3 + 4x^2 \)

q4844\\
\(\displaystyle \text{Which functions below are even? [Mark all that apply.]} \)

q4845r1\\
\(\displaystyle 4x - x^3 \)

q4845r2\\
\(\displaystyle x^5 - 5 \)

q4845r3\\
\(\displaystyle x \sin x \)

q4845r4\\
\(\displaystyle 4x^3 \)

q4845\\
\(\displaystyle \text{Which functions below are odd? [Mark all that apply.]} \)

q4846r1\\
\(\displaystyle \text{A} \)

q4846r2\\
\(\displaystyle \text{C} \)

q4846r3\\
\(\displaystyle \text{D} \)

q4846r4\\
\(\displaystyle \text{E} \)

q4846\\
\(\displaystyle f(x) = -x^2 + 4x - 2 \)

q4847r1\\
\(\displaystyle \text{A} \)

q4847r2\\
\(\displaystyle \text{B} \)

q4847r3\\
\(\displaystyle \text{C} \)

q4847r4\\
\(\displaystyle \text{F} \)

q4847\\
\(\displaystyle f(x) = 2x^2 - 3x - 2 \)

q4848r1\\
\(\displaystyle \text{A} \)

q4848r2\\
\(\displaystyle \text{B} \)

q4848r3\\
\(\displaystyle \text{C} \)

q4848r4\\
\(\displaystyle \text{E} \)

q4848\\
\(\displaystyle f(x) = x^2 + x - 2 \)

q4849r1\\
\(\displaystyle \text{A} \)

q4849r2\\
\(\displaystyle \text{B} \)

q4849r3\\
\(\displaystyle \text{E} \)

q4849r4\\
\(\displaystyle \text{F} \)

q4849\\
\(\displaystyle f(x) = x^4 + x^2 - 2 \)

q4850r1\\
\(\displaystyle f(x) = 2x^3 - 2x^2 \)

q4850r2\\
\(\displaystyle f(x) = x^3 + 5x \)

q4850r3\\
\(\displaystyle f(x) = 3x^3 + 2x^2 + x \)

q4850r4\\
\(\displaystyle f(x) = 2x^3 + x \)

q4850\\
\(\displaystyle \text{What function has the following graph?} \)

q4851r1\\
\(\displaystyle f(x) = e^x + x \)

q4851r2\\
\(\displaystyle f(x) = e^x + 1 \)

q4851r3\\
\(\displaystyle f(x) = 2e^x + 1 \)

q4851r4\\
\(\displaystyle f(x) = e^x - x + 1 \)

q4851\\
\(\displaystyle \text{What function has the following graph?} \)

q4852r1\\
\(\displaystyle (-\infty, 7^{1/2}), [-1, \infty) \)

q4852r2\\
\(\displaystyle (-\infty, 7) \cup [7, -1], [-1, \infty) \)

q4852r3\\
\(\displaystyle (-\infty, 7], [-1, \infty) \)

q4852r4\\
\(\displaystyle (-\infty, \infty), [-1, \infty) \)

q4852\\
\(\displaystyle \text{What are the domain and range, respectively, of } f(x) = (7-x)^{1/2} - 1? \)

q4853r1\\
\(\displaystyle \frac{2}{x+1} \)

q4853r2\\
\(\displaystyle \frac{1+x^2}{x} \)

q4853r3\\
\(\displaystyle \frac{x}{1 + x^2} \)

q4853r4\\
\(\displaystyle x + 1 \)

q4853\\
\(\displaystyle \text{If } f(x) = \frac{x^2 + 1}{x}, \text{ what is } f \left ( \frac{1}{x} \right ) ? \)

q4854r1\\
\(\displaystyle x \cos x \)

q4854r2\\
\(\displaystyle 2 \)

q4854r3\\
\(\displaystyle 3x^4 - 2x^2 \)

q4854r4\\
\(\displaystyle 2x^4 \cos x \)

q4854\\
\(\displaystyle \text{Which functions below are even? [Mark all that apply.]} \)

q4855r1\\
\(\displaystyle 4x^3 - 4 \)

q4855r2\\
\(\displaystyle x^9 + 3x^7 - 5x^5 + 7x^x - 9x \)

q4855r3\\
\(\displaystyle 19 \)

q4855r4\\
\(\displaystyle 5x^3 - x \)

q4855\\
\(\displaystyle \text{Which functions below are odd? [Mark all that apply.]} \)

q4856r1\\
\(\displaystyle \text{A} \)

q4856r2\\
\(\displaystyle \text{B} \)

q4856r3\\
\(\displaystyle \text{C} \)

q4856r4\\
\(\displaystyle \text{F} \)

q4856\\
\(\displaystyle f(x) = -\sin x - 1 \)

q4858r1\\
\(\displaystyle \text{C} \)

q4858r2\\
\(\displaystyle \text{D} \)

q4858r3\\
\(\displaystyle \text{E} \)

q4858r4\\
\(\displaystyle \text{F} \)

q4858\\
\(\displaystyle f(x) = -\cos x - 1 \)

q4859r1\\
\(\displaystyle \text{A} \)

q4859r2\\
\(\displaystyle \text{B} \)

q4859r3\\
\(\displaystyle \text{D} \)

q4859r4\\
\(\displaystyle \text{F} \)

q4859\\
\(\displaystyle f(x) = \cos (4x) \)

q4860r1\\
\(\displaystyle \text{A} \)

q4860r2\\
\(\displaystyle \text{C} \)

q4860r3\\
\(\displaystyle \text{D} \)

q4860r4\\
\(\displaystyle \text{E} \)

q4860\\
\(\displaystyle f(x) = x \cos (4x) \)

q4861r1\\
\(\displaystyle f(x) = x\ln x \)

q4861r2\\
\(\displaystyle f(x) = x \ln x^2 \)

q4861r3\\
\(\displaystyle f(x) = x^3 \)

q4861r4\\
\(\displaystyle f(x) = x^2 \ln 3x \)

q4861\\
\(\displaystyle \text{What function has the following graph?} \)

q4862r1\\
\(\displaystyle f(x) = \ln x \)

q4862r2\\
\(\displaystyle f(x) = \cos(\sin x) \)

q4862r3\\
\(\displaystyle f(x) = \sin(\ln x) \)

q4862r4\\
\(\displaystyle f(x) = \sin(\cos x) \)

q4862\\
\(\displaystyle \text{What function has the following graph?} \)

q4941r1\\
\(\displaystyle 0 \)

q4941r2\\
\(\displaystyle 1 \)

q4941r3\\
\(\displaystyle 3 \)

q4941r4\\
\(\displaystyle \text{Does not exist} \)

q4941\\
\(\displaystyle \text{Evaluate the limit } \lim_{x \rightarrow 2^-} f(x) \text{ by referring to the graph below.} \)

q4942r1\\
\(\displaystyle 0 \)

q4942r2\\
\(\displaystyle 1 \)

q4942r3\\
\(\displaystyle 3 \)

q4942r4\\
\(\displaystyle \text{Does not exist} \)

q4942\\
\(\displaystyle \text{Evaluate the limit } \lim_{x \rightarrow 2^+} f(x) \text{ by referring to the graph below.} \)

q4943r1\\
\(\displaystyle 0 \)

q4943r2\\
\(\displaystyle 1 \)

q4943r3\\
\(\displaystyle 3 \)

q4943r4\\
\(\displaystyle \text{Does not exist} \)

q4943\\
\(\displaystyle \text{Evaluate the limit } \lim_{x \rightarrow 2} f(x) \text{ by referring to the graph below.} \)

q4944r1\\
\(\displaystyle 0 \)

q4944r2\\
\(\displaystyle 1 \)

q4944r3\\
\(\displaystyle 3 \)

q4944r4\\
\(\displaystyle \text{Does not exist} \)

q4944\\
\(\displaystyle \text{Evaluate the limit } \lim_{x \rightarrow -\infty} f(x) \text{ by referring to the graph below.} \)

q4945r1\\
\(\displaystyle -3 \)

q4945r2\\
\(\displaystyle 0 \)

q4945r3\\
\(\displaystyle 4 \)

q4945r4\\
\(\displaystyle \text{Does not exist} \)

q4945\\
\(\displaystyle \text{Evaluate the limit } \lim_{x \rightarrow 1^-} f(x) \text{ by referring to the graph below.} \)

q4946r1\\
\(\displaystyle -3 \)

q4946r2\\
\(\displaystyle 0 \)

q4946r3\\
\(\displaystyle 4 \)

q4946r4\\
\(\displaystyle \text{Does not exist} \)

q4946\\
\(\displaystyle \text{Evaluate the limit } \lim_{x \rightarrow 1^+} f(x) \text{ by referring to the graph below.} \)

q4947r1\\
\(\displaystyle -3 \)

q4947r2\\
\(\displaystyle 0 \)

q4947r3\\
\(\displaystyle 4 \)

q4947r4\\
\(\displaystyle \text{Does not exist} \)

q4947\\
\(\displaystyle \text{Evaluate the limit } \lim_{x \rightarrow 1} f(x) \text{ by referring to the graph below.} \)

q4948r1\\
\(\displaystyle -3 \)

q4948r2\\
\(\displaystyle 0 \)

q4948r3\\
\(\displaystyle 4 \)

q4948r4\\
\(\displaystyle \text{Does not exist} \)

q4948\\
\(\displaystyle \text{Evaluate the limit } \lim_{x \rightarrow \infty} f(x) \text{ by referring to the graph below.} \)

q4411\\
\(\displaystyle \text{Let } f(x) = x^3 - 3x^2/2 - 6x + 3/2. \text{ Solve each of the problems below. In problems 1-4, mark all answers that apply.} \)

q4412\\
\(\displaystyle \text{Let } f(x) = x^4 / 4 + x^3 + 3x^2 / 2 + x + 1/4. \text{Solve each of the problems below. In problems 1-4, mark all answers that apply.} \)

q4413\\
\(\displaystyle \text{Let } f(x) = 4x^3 / 3 - x + 1/3. \text{Solve each of the problems below. In problems 1-4, mark all answers that apply.} \)

q4512\\
\(\displaystyle \text{Let } f(x) = 2x^3 - 3x^2 - 12x + 5. \text{Solve each of the problems below. In problems 1-4, mark all answers that apply.} \)

q4513\\
\(\displaystyle \text{Let } f(x) = 4x^3 - 6x^2 - 9x. \text{ Solve each of the problems below. In problems 1-4, mark all answers that apply.} \)

q4516\\
\(\displaystyle \text{Let } f(x) = x^3 + 2x^2 - 4x + 4. \)

q4517\\
\(\displaystyle \text{Let } f(x) = 2 + 9x - x^3. \)

q4518\\
\(\displaystyle \text{Let } f(x) = x^4 - 24x^2. \)

q4519\\
\(\displaystyle \text{Let } f(x) = 4x^3 - 15x^2 - 18x + 10. \)

